\documentclass[../../topologia_algebraica]{subfiles}
\begin{document}
\chapter*{Categor\'ias}

Para probar el teorema de la invariancia homot\'opica de la homolog\'ia singular fue la
existencia del diagrama conmutativo
\[
  \begin{tikzcd}
    S_n(X) \arrow[r,"T^X_n"] \arrow[d,"f_{\#}"'] & S_{n+1}(X\times I) \arrow[d,"(f\times\Id_I)_{\#}"]\\
    S_n(Y) \arrow[r,"T^Y_n"'] & S_{n+1}(Y\times I)
  \end{tikzcd}
\]
Para probar el teorema de esici\'on nos apoyamos en la sucesi\'on exacta larga de la
homolog\'ia relativa junto con el diagrama conmutativo
\[
  \begin{tikzcd}[column sep=large]
    \cdots \arrow[r] & H_n(A) \arrow[r,"H_n(\imath)"] \arrow[d,"H_n(f|_A)"]&
    H_n(X) \arrow[r,"H_n(j)"] \arrow[d,"H_n(f)"] & H_n(X,A) \arrow[r,"d_n"] \arrow[d,"H_n(f)"] &
    H_{n-1}(A) \arrow[r,"H_{n-1}(\imath)"] \arrow[d,"H_{n-1}(f|_A)"] &
    H_{n-1}(X) \arrow[r] \arrow[d,"H_{n-1}(f)"] & \cdots\\
     \cdots \arrow[r] & H_n(B) \arrow[r,"H_n(\imath')"'] & H_n(X) \arrow[r,"H_n(j')"'] &
    H_n(Y,B) \arrow[r,"d'_n"'] & H_{n-1}(B) \arrow[r,"H_{n-1}(\imath')"'] & H_{n-1}(Y) \arrow[r] & \cdots
   \end{tikzcd}
 \]
 Estas consideraciones llevaron a Eilenberg y a Maclane a crear la teor\'ia de categor\'ias.

 \begin{defin}
   Una \emph{categor\'ia} $\Cc$ consiste de tres cosas:
   \begin{enumerate}
   \item Una clase $\obj{\Cc}$ de objetos.
   \item Para cada dos objetos $A,B\in\obj{\Cc}$, existe un conjunto $\Hom{A,B}$ cuyos
     elementos se llaman morfismos y se denotan $f:A\ra B$ o $A\morf{f}B$.
   \item Para cada terna $A,B,C\in\obj{\Cc}$ existe una funci\'on
     \[
       \circ:\Hom{A,B}\times\Hom{B,C}\lra \Hom{A,C} \quad\text{definido por}\quad
       (f,g)\mapsto g\circ f,
     \]
     llamado \emph{composici\'on} que cumple las siguientes dos propiedades:
     \begin{enumerate}
     \item $\circ$ es asociativa, es decir $(h\circ g)\circ f=h\circ(g\circ f)$.
     \item Para todo objeto $A\in\obj{\Cc}$ existe un morfismo $\Id_A\in\Hom{A,A}$
       tal que para todo objeto $B\in\obj{\Cc}$, y cualesquiera $f\in\Hom{A,B}$ y
       $g\in\Hom{B,A}$, se cumple
       \[
         f\circ\Id_A=f\quad\text{y}\quad \Id_A\circ g=g
       \]
     \end{enumerate}
   \end{enumerate}
 \end{defin}

 \begin{nota}
   Usualmente omitimos la notaci\'on $\circ$ para simplemente escribir $gf$ en lugar de $g\circ f$.
   Tambi\'en pedimos que, si $A\neq C$ o $B\neq D$ entonces $\Hom{A,B}\cap\Hom{C,D}=\emptyset$.
   Es decir los morfismos est\'an determinados, en parte, por su dominio y su contradominio.
 \end{nota}

 Las matem\'aticas est\'an llenas de categor\'ias:

 \begin{ejemplo}$\;$\\
   \begin{enumerate}
   \item $\cat{Top}$: $\obj{\cat{Top}}=\{\text{espacios topol\'ogicos}\}$,
     $\Hom{X,Y}=\{f:X\ra Y\mid f\;\text{es continua}\}$.
   \item $\cat{Top}_*$: $\obj{\cat{Top}_*}=\{(X,x)\mid X\in\obj{\cat{Top}},x\in X\}$,
     $\Hom{(X,x),(Y,y)}=\{f:X\ra Y\mid f\;\text{es continua y}\;f(x)=y\}$.
   \item $\cat{Top}_2$: $\obj{\cat{Top}_2}=\{(X,A)\mid X,A\in\obj{\cat{Top}},A\subseteq X\}$,
     $\Hom{(X,A),(Y,B)}=\{f:X\ra Y\mid f\;\text{es continua y}\; f[A]\subseteq B\}$.
   \item $\cat{Var}$: $\obj{\cat{Var}}=\{\text{variedades suaves}\}$,
     $\Hom{M,N}=\{f:M\ra N\mid f\;\text{es suave}\}$.
   \item $h\cat{Top}$: $\obj{h\cat{Top}}=\obj{\cat{Top}}$,
     $\Hom{X,Y}=[X,Y]$ donde la composici\'on est\'a definida por $[g][f]=[g\circ f]$
     (cf. definici\'on \ref{def:map_modulo_homotopia}).
   \item $\cat{Grupos}$: $\obj{\cat{Grupos}}=\{\text{grupos}\}$,
     $\Hom{G,H}=\{f:G\ra H\mid f\;\text{es un homomorfismo de grupos}\}$. 
   \item $\cat{Ab}$: $\obj{\cat{Ab}}=\{\text{grupos abelianos}\}$,
     $\Hom{G,H}=\{f:X\ra Y\mid f\;\text{es homomorfismo de grupos}\}$.
   \item ${}_R\cat{Mod}$: $\obj{{}_R\cat{Mod}}=\{R\text{-m\'odulos izquierdos}\}$,
     $\Hom{M,N}=\{f:M\ra N\mid f\;\text{es un morfismo de}\;R\text{-m\'odulos}\}$.
   \item $\cat{CompSimp}$: $\obj{\cat{CompSimp}}=\{\text{complejos simpliciales}\}$,
     $\Hom{K,L}=\{f:K\ra L\mid f\;\text{es un mapea simplicial}\}$.
   \item $\cat{Comp}(\Aa)$: $\obj{\cat{Comp}(\Aa)}=\{\text{complejos de cadena en la categor\'ia}\;\Aa\}$,
     $\Hom{\Cc_{\bullet},C'_{\bullet}}=\{f_{\bullet}:\Cc_{\bullet}\ra \Cc'_{\bullet}\mid%
     f_{\bullet}\;\text{es un morfismo de complejos de cadena}\}$.
   \item $\cat{Cov}(X)$: $\obj{\cat{Cov}(X)}=\{\text{cubrientes sobre}\; X\}$,
     $\Hom{(E,p,X),(E',p',X)}=\{f:E\ra E'\mid f\;\text{es continua y}\; f\circ p'=p\}$.
   \item Si $(X,\leq)$ es un conjunto parcialmente ordenado, entonces tiene naturalmente
     la estructura de una categor\'ia: define $\obj{(X,\leq)}=X$ y
     \[
       \Hom{x,x'}=
       \begin{cases}
         \imath^x_{x'} &\text{si}\;\; x\leq x' \\
         \emptyset &\text{si}\;\; x\not\leq x'.
       \end{cases}
     \]
   \item Todo grupo $G$ se puede realizar como una categor\'ia: define $\obj{G}=\{\bullet\}$
     y $\Hom{\bullet,\bullet}=G$ donde la composici\'on est\'a dada por la multiplicaci\'on
     del grupo, ie. $g\circ g'=gg'$.
   \end{enumerate}
 \end{ejemplo}

 \begin{defin}
   Una categor\'ia $\Cc$ es \emph{peque\~na} si $\obj{\Cc}$ es un conjunto.
 \end{defin}

 Una definic\'on que siempre hemos usado es la de isomorfismo:

 \begin{defin}
   Sea $\Aa$ una categor\'ia y $A,B\in\obj{\Aa}$. Un morfismo $f\in\Hom{A,B}$ es
   un \emph{isomorfismo} si existe un morfismo $g\in\Hom{B,A}$ tal que $gf=\Id_A$
   y $fg=\Id_B$.
 \end{defin}

 Otra definici\'on importante es la de funtor:
 
 \begin{defin}
   Sean $\Aa$ y $\Bb$ categor\'ias. Un \emph{funtor} (covariante), denotado por $\Ff:\Aa\ra\Bb$,
   es una asignaci\'on:
   \[
     A \mapsto \Ff(A)\in\obj{\Bb} \quad\text{y}\quad
     \Big(A\morf{f} A' \Big) \mapsto \Big(\Ff(A)\morf{\Ff(f)} \Ff(A') \Big)
   \]
   que cumple las siguiente dos propiedades:
   \begin{enumerate}
   \item $\Ff(\Id_A)=\Id_{\Ff(A)}$
   \item $\Ff(gf)=\Ff(g)\Ff(f)$
   \end{enumerate}
 \end{defin}

 \begin{ejemplo}$\;$\\
   \begin{enumerate}
   \item $\pi_n:\cat{Top}_*\ra\cat{Grupos}$ con $(X,x_0)\mapsto\pi_n(X,x_0)$.
   \item $H_n(\_;R):\cat{Top}\ra{}_R\cat{Mod}$ con $X\mapsto H_n(X;R)$.
     \item $\abs{\cdot}:\cat{CompSimp}\ra\cat{Top}$ con $K\mapsto\abs{K}$.
   \end{enumerate}
 \end{ejemplo}

 Los funtores preservan isomorfismos:

 \import{\directory}{ejercicios/68} %%%%%%%%%%%%%%%%%%%%%%%%%%%%%%%%%%%%%%%%%%% EJERCICIO 68

 \import{\directory}{ejercicios/69} %%%%%%%%%%%%%%%%%%%%%%%%%%%%%%%%%%%%%%%%%%% EJERCICIO 69

 Mientras que la definici\'on de funtor es importante, es m\'as interesante la
 de transformaci\'on natural:

 \begin{defin}
   Sean $\Aa$ y $\Bb$ categor\'ias, $\Ff,\Gg:\Aa\ra\Bb$ funtores. Una \emph{transformaci\'on
     natural} entre $\Ff$ y $\Gg$, denotado por $\fT:\Ff\ra\Gg$ es una familia de morfismos
   $\fT=\{T_A:\Ff(A)\ra\Gg(A)\}_{A\in\obj{\Aa}}$ tales que hacen conmutar el siguiente diagrama:
   \[
     \begin{tikzcd}
       \Ff(A) \arrow[r,"T_A"] \arrow[d,"\Ff(f)"'] & \Gg(A) \arrow[d,"\Gg(f)"] \\
       \Ff(A') \arrow[r,"T_{A'}"'] & \Gg(A')
     \end{tikzcd}\quad\forall A,A'\in\obj{\Aa},\;\;\forall f\in\Hom{A,A'}.
   \]
   Si adem\'as cada $T_A\in \fT$ es un isomorfismo, decimos que $\fT$ es una \emph{equivalencia
   natural}.
\end{defin}

\begin{ejemplo}
  \begin{enumerate}
  \item Sea $S_{n}(\_;R):\cat{Top}\ra{}_R\cat{Mod}$ el funtor
    $X\mapsto S_{n}(X;R)$ y $f\mapsto f_{n}$, la $n$-\'esima componente del morfismo
    $f_{\#}:S_{\bullet}(X)\ra S_{\bullet}(Y)$ de complejos de cadena. Sea $\fI:\cat{Top}\ra\cat{Top}$
    el funtor $X\mapsto X\times I$ con $f\mapsto f\times\Id$. Definimos:
    \[
      \Ff=(S_{n+1}(\_;R)\circ\fI):\cat{Top}\lra {}_R\cat{Mod} \quad\text{con}\quad
      X\mapsto S_{n+1}(X\times I)\;\;,\;\; f\mapsto (f\times\Id)_{\#}
    \]
    
    Observa que $\fT=\{T_A:S_n(X)\ra S_{n+1}(X\times I)\}_{X}$ es una transformaci\'on natural
    (cf. secci\'on \ref{sec:invariancia}).
  \item Tomamos el funtor $H_n(\_,\_;R)$ el funtor de homolog\'ia relativa. Define
    $\Ff_n=H_{n-1}(\_;R)\circ\pi$ donde $\pi:\cat{Top}_2\ra\cat{Top}$ es el funtor
    $(X,A)\mapsto A$ y $f\mapsto f|_{A}$. En s\'imbolos:
    \[
      \Ff_n:\cat{Top}_2\lra {}_R\cat{Mod} \quad\text{con}\quad (X,A)\mapsto H_{n-1}(A;R)
    \]
    Entonces la familia de morfismos de conexi\'on $\fD=\{d_n^{(X,A)}:H_n(X,A;R)\ra H_{n-1}(A)\}_{(X,A)}$
    es una transformaci\'on natural entre $H_n(\_,\_;R)$ y $\Ff_n$.
  \item Sea $\Vv_{<\infty}$ la categor\'ia de $k$-espacios vectoriales de dimensi\'on finita con
  transformaciones lineales como morfismos. Sea $\Ii:\Vv_{<\infty}\ra\Vv_{<\infty}$ el funtor
  identidad, es decir $\Ii(V)=V$ y $\Ii(f)=f$. Ahora define $F$ cumple el funtor ``doble dual'',
  es decir $\Ff:\Vv_{<\infty}\ra\Vv_{<\infty}$ con $V\mapsto V^{**}$ y $\Ff(f)$ definido de la
  manera can\'onica. M\'as precisamente, si $f:V\ra W$ es una transformaci\'on lineal y
  $\alpha:V^*\ra k$ es un elemento de $V^{**}=\Hom{V^*,k}$, entonces $\Ff(f)(\alpha)$ se define
  como la funci\'on $\Ff(f)(\alpha):V^{**}\ra W^{**}$ que hace
  $\Ff(f)(\alpha)(\beta)=\alpha(\beta\circ f)$, donde $\beta:W\ra k$.
    
  \end{enumerate}
\end{ejemplo}
 \import{\directory}{ejercicios/70} %%%%%%%%%%%%%%%%%%%%%%%%%%%%%%%%%%%%%%%%%%% EJERCICIO 70


\end{document}