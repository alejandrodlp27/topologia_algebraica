\documentclass[../../topologia_algebraica]{subfiles}
\begin{document}
\section{Generalizaci\'on del grupo fundamental a otras dimensiones}

Para poder generalizar el grupo fundamental a otras dimensiones, necesito encontrar otra definici\'on
equivalente de la cual resulte obvio generalizar a otras dimensions.
Para este fin, considera el siguiente argumento:

Un lazo $\alpha\in\Omega(X,x_0)$ es una funci\'on continua $\alpha:I\ra X$ tal que
$\alpha(0)=x_0=\alpha(1)$. Esto significa que $\alpha$ se factoriza a trav\'es
de $I/\partial I$. M\'as precisamente:

La funci\'on $\nu I\ra \Sn^1$ definido por $\nu(t)=e^{2\pi i t}$ es continua y sobre. Adem\'as,
para toda $U\subseteq\Sn^1$ se tiene
\[
  U\;\;\text{es abierto} \quad\iff\quad \nu^{-1}[U]\;\; \text{es abierto}.
\]
Por lo tanto $\nu$ es una proyecci\'on de espacios topol\'ogicos.

Esto quiere decir que $\Sn^1$ tiene la topolog\'ia cociente: sobre $I$ definimos la siguiente
relaci\'on de equivalencia:
\[
  s\sim t \quad\iff\quad \nu(s)=\nu(t).
\]
Esta relaci\'on parte el conjunto en todos los singuletes $\{s\}_{s\neq0,1}$ y en $\partial I=\{0,1\}$.
Por lo tanto el espacio de clases cumple
\[
  \frac{I}{\partial I}\approx \Sn^1.
\]

Adem\'as, cualquier funci\'on continua $\alpha:I \ra X$ tal que $\alpha(0)=\alpha(1)$, ie. un lazo
se factoriza a trav\'es de $\nu$:
\[
  \begin{tikzcd}
   I \arrow[r,"\alpha"] \arrow[d,twoheadrightarrow,"\nu"'] & X \\
   \frac{I}{\partial I} \arrow[ur,dotted,"\tilde{\alpha}"'] & 
 \end{tikzcd}
 \quad\text{con}\quad \alpha=\tilde{\alpha}\circ\nu
\]
Por lo tanto hay una biyecci\'on natural

\[
  \begin{tikzcd}
    \Big[(\Sn^1,1),(X,x_0)\Big] \arrow[r,"\Phi"] & \pi_1(X,x_0) & \text{con}
    & \text{[}\beta\text{]} \arrow[r,mapsto] & \text{[}\beta\circ\nu\text{]}
  \end{tikzcd}
\]

con inverso $[\alpha]\mapsto[\tilde{\alpha}]$. Est\'a bien definido por la proposici\'on
\ref{prop:homotopia_composicion}: $\alpha\simeq\beta\;\;\then\;\; \alpha\circ\nu \simeq \beta\circ\nu$.

Adem\'as, esta biyecci\'on es un homeomorfismo...................................................

\begin{prop}
  Si $\Omega(X,x_0)$ y $\text{Map}_*((\Sn^1,1),(X,x_0))$ tienen la topolog\'ia compacto-abierta;
  si $\pi_1(X,x_0)$ y $[(\Sn^1,1),(X,x_0)]$ tienen sus correspondientes topolog\'ias cociente, entonces
  \[
    \Big[(\Sn^1,1),(X,x_0)\Big] \approx \pi_1(X,x_0)
  \]
\end{prop}
\begin{proof}
  Primero pruebo que $\Phi$ es continuo. Sea $U\subseteq\pi_1(X,x_0)$, entonces su preimagen $\bar{U}$
  en
  \[
    \Omega(X,x_0)=\{I\morf{\alpha} X \mid \alpha \;\text{es continua y}\; \alpha(0)=\alpha(1)\},
  \]
  con la topolog\'ia de subespacio de $C^0(I,Y)$, es abierto, es decir es una uni\'on arbitraria
  de intersecciones finitas de los abiertos $B_K(U)$:
  \[
    \bar{U}=\bigcup_{j\in J}B_{K_j}(U_j)
  \]
  con $K_j\subseteq I$ compacto y $U_j\subseteq Y$ abierto. Quiero probar que la preimagen bajo
  $\Phi$ de $\bar{U}$ es un abierto. Como $\Phi$ es biyectiva, basta probar esto para $\bar{U}=B_K(U)$
  (esto es porque la)..
  
\end{proof}

\begin{nota}
  Puedo intercambiar $(\Sn^1,1)$ por $(I/\partial I,\star)$ donde $\star\in I/\partial I$ es la clase
  de equivalencia $[\partial I]$ ya que son homeomorfos como espacios basados.
\end{nota}

Este resultado motiva la siguiente definici\'on:
\begin{defin}\label{def:grupo_fundamental_dimension_n}
  El grupo fundamental de dimensi\'on $n$ de un espacio basado $(X,x_0)$ se define como
  \[
    \pi_n(X,x_0):=\Big[(\Sn^n,1),(X,x_0)\Big].
  \]
\end{defin}

Primero analizamos qu\'e sucede cuando $n=0$ ya que este caso es distinto a los dem\'as.

Recuerda que $\Sn^0=\{x\in\RR: \abs{x}=1\}=\{-1,1\}$. Llamar\'e a estos puntos de otra manera
para que las cuentas sean m\'as n\'itidas: sean $1=\star$ y $-1=\bullet$. Con esta notaci\'on
puedo escribir:
\[
  M:=\text{Map}_*\Big( (\Sn^0,\star),(X,x_0) \Big)=
  \{\Sn^0\morf{f} X \mid f\;\text{es continua y}\; f(\star)=x_0\}.
\]
Entonces la \'unica informaci\'on que nos falta saber para determinar a $f$ es el valor
$f(\bullet)\in X$. Esto nos induce la funci\'on:
\[
  \begin{tikzcd}
    M\arrow[r,"\Psi"] & X & \text{con}
    & f \arrow[r,mapsto] & f(\bullet).
  \end{tikzcd}
\]
Observa que si $f,g\in M$ son homot\'opicos, existe una funci\'on continua
$H:\{\star,\bullet\}\times I\ra X$ donde $H_o=f$ y $H_1=g$. Esto significa que
la funci\'on $F_{\bullet}:I\ra X$ definida por $F_{\bullet}(t)=H(\bullet,t)$, es continua.
Adem\'as $F_{\bullet}(0)=H_0(\bullet)=f(\bullet)=\Psi(f)$ y $F_{\bullet}(1)=H_1(\bullet)=g(\bullet)=\Psi(g)$.
Es decir $F_{\bullet}$ es una trayectoria de $\Psi(f)$ a $\Psi(g)$ en $X$. Decimos
que $\Psi(f)$ y $\Psi(g)$ son \emph{conectables por trayectorias}.

Con esto podemos definir una relaci\'on de equivalencia natural en $X$: para todas $x,y\in X$
\begin{align*}%
  x\sim y & \quad\iff\quad x \;\text{y} y \;\text{son conectables por trayectorias}\\ &%
  \quad\iff\quad%
  \exists\;\; \sigma:I\ra X \;\text{continua tal que}\; \sigma(0)=x \;\text{y}\; \sigma(1)=y.%
\end{align*}%
Por lo tanto, el argumento del p\'arrafo anterior nos dice que:
\[
  f\simeq g \quad\then\quad \Psi(f)\sim \Psi(g)
\]
y as\'i, puedo definir $\Psi$ para clases de equivalencia:
\[
  \Psi\big( [f]_{\simeq} \big)= [\Psi(f)]_{\sim} \quad\text{est\'a bien definida}.
\]
Deber\'ia de cambiar de notaci\'on a algo como $\hat{\Psi}$, pero no creo que cause problemas.
Observa que $[f]_{\simeq}\in M/_{\simeq}$ que es precisamente $\pi_0(X,x_0)$. Por lo tanto
$\Psi:\pi_0(X,x_0)\ra X/_{\sim}$.

Pruebo que $\Psi$ es biyectiva:

Para toda $[x]_{\sim}\in X/_{\sim}$, toma $f_x\in M$ definida por $f(\bullet)=x$. Claramente
\[
  \Psi([f_x]_{\simeq})=[\Psi(f)]_{\sim}=[f(\bullet)]=[x]_{\sim},
\]
y as\'i $\Psi$ es sobreyectiva.

Sean $[\alpha],[\beta]\in\pi_0(X,x_0)$ tales que  que $\Psi([\alpha])=Psi([\alpha])$, es decir que
$\alpha(\bullet)$ y $\beta(\bullet)$ son conectables por trayectorias; supongamos que es mediante
la trayectoria $\sigma:I\ra X$. Defino la siguiente homotop\'ia entre $\alpha$ y $\beta$:
\[
  H(s,t):=
  \begin{cases}
    x_0 & \text{si}\;\; s=\star \\
    \sigma(t) & \text{si}\;\; s=\bullet
  \end{cases}
\]
Esta es una homotop\'ia porque es la uni\'on disjunta de funciones continuas y es una calca de
la homotop\'ia $F$ que constru\'i para definir la relaci\'on de equivalencia en $X$. Por lo
tanto $\alpha\simeq\beta$, entonces $[\alpha]=[\beta]$ y concluyo que $\Psi$ es inyectiva.

Por \'ultimo observa que $X/_{\sim}$ se puede ver como el espacio de componentes conexas
del espacio $X$ ya que al hacer cociente reducimos toda una componente conexa (arco-conexa)
a un punto sobre ella.

Con todo esto he probado que
\begin{prop}\label{pi_cero_componentes}
  \[
  \begin{tikzcd}
    \pi_0(X,x_0) \arrow[r,"\Psi"] &
    \Big\{  \begin{smallmatrix}\text{componentes}\\ \text{conexas de}\; X\end{smallmatrix}  \Big\}&
    \text{es biyectiva.}
  \end{tikzcd}
\]
\end{prop}


Los dem\'as grupos fundamentales son muy parecidos a $\pi_1(X,x_0)$. La \'unica diferencia es que
los lazos ahora est\'an definidos sobre $I^n/\partial I^n$ en lugar de sobre $I/\partial I$.
Si defino la operaci\'on como:
\[
  (\alpha*\beta)(s_1,\ldots,s_n):=
  \begin{cases}
    \alpha(2s_1,s_2,\ldots,s_n) & \text{si}\;\; 0\leq s_1\leq\frac{1}{2} \\
    \alpha(2s_1-1,s_2,\ldots,s_n) & \text{si}\;\; \frac{1}{2}\leq s_1 \leq 1
  \end{cases}
\]
donde $\alpha,\beta:I^n\ra X$, entonces todas las propiedades de grupo de $\pi_n(X,x_0)$
son id\'enticas a las propiedades de $\pi_1(X,x_0)$. Las enumero para tenerlas a la mano:
\begin{align*}%
  \big(\pi_n(X,x_0),*\big ) &  \quad\text{es un grupo} \\ %
  [e]=[e_{x_0}]\quad & \text{es el neutro} \\  %
  \forall\;\; [\alpha],\quad &  [\alpha]^{-1}=[\bar{\alpha}]
\end{align*}%

Tambi\'en observa que 
  \[
    \begin{tikzcd}
      (X,x_0) \arrow[r,mapsto,"\Ff"] & \pi_n(X,x_0) &
      \Big\{ (X,x_0)\morf{f}(Y,y_0) \Big\} \arrow[r,mapsto] &
      \Big\{ \pi_n(X,x_0)\morf{f_{\#}}\pi_n(Y,y_0) \Big\}
    \end{tikzcd}
  \]
  es un funtor (covariante) de la categor\'ia $\mathbf{Top}_*$ a la categor\'ia de grupos. La
  prueba de esto es la misma que en el Teorema \ref{thm:grupo_fundamental_funtor}.
\end{document}