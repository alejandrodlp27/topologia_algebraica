\documentclass[../../topologia_algebraica]{subfiles}
\begin{document}
\section{El funtor $(X,x_0) \mapsto \pi_1(X,x_0)$}

Como $\pi_1(X,x_0)$ es un grupo, pertenece como objeto a la categor\'ia $\mathbf{Grupos}$ cuyos
objetos son grupos y cuyos morfismos son los homomorfismos de grupo. Sea antoja que
\[
  \Ff:\mathbf{Top}_* \lra \mathbf{Grupos} \quad\text{con}\quad \Ff(X,x_0)=\pi_1(X,x_0)
\]
sea un funtor. Para esto debemos definir un homomorfismo de grupos $\Ff(f)$ para cada morfismo
de espacios basados $f$.

Sea $f\in\Hom{(X,x_0),(Y,y_0)}$, es decir $f:X\ra Y$ es continua y $f(x_0)=y_0$. Entonces si
$\alpha\in\Omega(X,x_0)$ es un lazo, tendremos que $f\circ\alpha$ es un lazo en $(Y,y_0)$. En
efecto, $f\circ\alpha$ es continua y
$(f\circ\alpha)(0)=f(\alpha(0))=f(x_0)=f(\alpha(1))(f\circ\alpha)(1)$. Por lo tanto
$a\mapsto f\circ\alpha$ es una funci\'on bien definida entre $\Omega(X,x_0)$ y $\Omega(Y,y_0)$.

Para ver que esa funci\'on induce una funci\'on entre los grupos fundamentales, hay que
probar que la funci\'on respeta homotop\'ias, es decir
\[
  \alpha\simeq\beta \quad\then\quad (f\circ\alpha)\simeq(f\circ\beta).
\]
Esto es consecuencia directa de la proposici\'on \ref{prop:homotopia_composicion}. Por lo tanto
\[
  f_{\#}:\pi_1(X,x_0) \lra \pi_1(Y,y_0) \quad\text{con}\quad f_{\#}\big([\alpha]\big)=[f\circ\alpha]
\]
es una funci\'on bien definida. Observa que:
\[
  f_{\#}\big([\alpha][\beta]\big)=f_{\#}\big( [\alpha*\beta] \big)=\big[ f\circ(\alpha*\beta) \big]=
  \big[(f\circ\alpha)*(f\circ\beta)\big]=[f\circ\alpha][f\circ\beta]=
  f_{\#}\big([\alpha]\big)f_{\#}\big([\beta]\big)
\]
donde la tercera igualdad se da porque los lazos son iguales:
\begin{align*}
  \big(f\circ(\alpha*\beta) \big)(s) & =
  \begin{cases}
    f(\alpha)(2s)=f_{\#}(\alpha)(2s) & \text{si}\;\; 0\leq s\leq \frac{1}{2}\\
    f(\beta)(2s-1)=f_{\#}(\beta)(2s-1) & \text{si}\;\; \frac{1}{2}\leq s\leq 1
  \end{cases} \\ & =
  \Big(f_{\#}(\alpha)*f_{\#}(\beta)\Big)(s)=\Big((f\circ\alpha)*(f\circ\beta)\Big).
\end{align*}
Adem\'as $f_{\#}([e_{x_0}])=[f\circ e_{x_0}]=[e_{y_0}]$ porque $f(x_0)=y_0$.

Por lo tanto $f_{\#}$ es un homomorfismo de grupos y as\'i $\Ff$ est\'a bien definido
para ser funtor, s\'olo hacen falta dos propiedades.

Claramente
\[
  (\Id_X)_{\#}\big( [\alpha] \big)=[\Id_X\circ\alpha]=[\alpha],
\]
entonces $(\Id_X)_{\#}$ es la funci\'on identidad en $\pi_1(X,x_0)$. Por \'ultimo, si
$f:(X,x_0)\ra (Y,y_0)$ y $g:(Y,y_0)\ra(Z,z_0)$ son morfismos de espacios basados, entonces 
\[
  (g\circ f)_{\#}\big( [\alpha] \big)= \big[ (g\circ f) \circ\alpha \big]=
  \big[ g\circ (f \circ\alpha) \big]=g_{\#}\Big( [f\circ\alpha] \Big)=
  g_{\#}\Big( f_{\#}\big([\alpha]\big) \Big)=(g_{\#}\circ f_{\#})\big( [\alpha] \big)
\]
y as\'i concluyo que
\[
  (g\circ f)_{\#}=g_{\#}\circ f_{\#}
\]

Todo esto prueba:
\begin{thm}\label{thm:grupo_fundamental_funtor}
  La asignaci\'on $\Ff:\mathbf{Top}_*\ra \mathbf{Grupos}$ definido por
  \[
    \begin{tikzcd}
      (X,x_0) \arrow[r,mapsto,"\Ff"] & \pi_1(X,x_0) &
      \Big\{ (X,x_0)\morf{f}(Y,y_0) \Big\} \arrow[r,mapsto] &
      \Big\{ \pi_1(X,x_0)\morf{f_{\#}}\pi_1(Y,y_0) \Big\}
    \end{tikzcd}
  \]
  es un funtor (covariante).
\end{thm}

La importancia de este teorema es que ahora tenemos una forma de estudiar los grupos fundamentales.
El teorema introduce un concepto importante para clasificar espacios:

\begin{cor}
  El grupo fundamental de un espacio basado es un \emph{invariante topol\'ogico}, es decir, si
  $(X,x_0)$ y $(Y,y_0)$ son espacios basados, entonces:
  \[
    (X,x_0)\approx(Y,y_0) \quad\then\quad \pi_1(X,x_0)\cong \pi_1(Y,y_0) 
  \]
\end{cor}
\begin{proof}
  Los isomorfismos de una categor\'ia se preservan bajo funtores, m\'as precisamente si
  $f:(X,x_0)\ra(Y,y_0)$ es un isomorfismo en $\mathbf{Top}_*$ (ie. $f:X\ra Y$ es un homeomorfismo
  y $f(x_0)=y_0$), entonces $\Ff(f)=f_{\#}:\pi_1(X,x_0)\ra\pi_1(Y,y_0)$ es un ismorfismo.
\end{proof}

\begin{nota}
  Este corolario nos da un criterio \'util para saber cuando dos espacios basados no son homeomorfos:
  simplemente niega la implicaci\'on del corolario para escribir
  \[
    \pi_1(X,x_0)\not\cong \pi_1(Y,y_0) \quad\then\quad (X,x_0)\not\approx(Y,y_0).
  \]
\end{nota}

Observa que el funtor del grupo fundamental toma como objetos espacio basados, no s\'olo espacios
topol\'ogicos. Esto nos restringe un poco la informaci\'on que podemos deducir al calcular un
grupo fundamental de un espacio. Esto se nota especialemtente en el caso $(X,x_0)\neq(X,x_1)$ ya que
los espacios topol\'ogicos pueden ser el mismo. Por suerte hay un (pseudo) remedio para esto:

\import{\directory}{ejercicios/5} %%%%%%%%%%%%%%%%%%%%%%%%%%%%%%%%%%%%%%%%%%%%%%%%%%%%%% EJERCICIO 5

El funtor $\mathfrak{F}$ preserva productos:

\begin{prop}
  Sea $\{(X_j,x_j)\}_{j\in J}$ una familia de espacios basados, entonces las proyecciones
  $p_i:\Pi (X_j,x_j) \ra (X_i,x_i)$ inducen un isomorfismo natural
  \[
    \pi_n\Big(\prod_{j\in J}(X_j,x_j),\star\Big) \cong \prod_{j\in J}\pi_n(X_j,x_j)
  \]
  donde $\star=\{x_j\}_{j\in J}$ es el punto base can\'onico de $\Pi (X_j,x_j)$.
\end{prop}
\begin{proof}
  Aplico el funtor $\mathfrak{F}$ a la familia $\{p_i:\Pi (X_j,x_j)\ra (X_i,x_i\}_{i\in J}$ para obtener
  una familia de homomorfismos de grupos
  \[
    \Big\{ (p_i)_{\#}: \pi_n \Big(\prod (X_j,x_j) ,\star  \Big) \lra \pi_n(X_i,x_i) \Big\}_{i\in J}.
  \]
  Por la propiedad universal del producto en $\cat{Grupos}$ existe un \'unico homomorfismo
  $\Phi:\pi_n(\Pi (X_j,x_j)\ra(X_i,x_i))\ra\Pi\pi_n(X_i,x_i)$ a trav\'es del cual se factorizan todas
  las proyecciones $(p_i)_{\#}$:
  \[
    \begin{tikzcd}
      \big(\prod (X_j,x_j),\star\big) \arrow[d,"p_i"'] \arrow[r,"\pi_n"] &
      \pi_n\big(\prod (X_j,x_j),\star\big) \arrow[d,"(p_i)_{\#}"'] \arrow[r,dashed,"\Phi"] &
      \prod \pi_n(X_j,x_j) \arrow[ld,"q_i"] \\
      (X_i,x_i) \arrow[r,"\pi_n"] &
      \pi_n(X_i,x_i) & 
    \end{tikzcd} \quad q_i\circ\Phi =(p_i)_{\#}
  \]
  donde las $q_i$'s son las proyecciones naturales del producto en $\cat{Grupos}$. El homomorfismo
  $\Phi$ es un isomorfismo:
  \begin{enumerate}
  \item[($\Phi$ es sobre)] Sea $\alpha=\{[\alpha_j]\}_{j\in J}\in \Pi \pi_n(X_j,x_j)$ un elemento
    arbitrario. Claramente induce una familia de funciones continuas $\{\alpha_j:\Sn^n\ra X_j\}$.
    En $\cat{Top}$, el producto cartesiano es el producto, entonces existe un \'unico morfismo
    $\widetilde{\alpha}:\Sn^n \ra \Pi X_j$ que hace conmutar el siguiente diagrama:
    \[
      \begin{tikzcd}
        \Sn^n \arrow[rr,dashed,"\widetilde{\alpha}"] \arrow[dr,"\alpha_i"'] &&
        \prod X_j \arrow[dl,"p_i"] \\
        & X_i  &
      \end{tikzcd}
    \]

    Observa que $[\widetilde{\alpha}]\in\pi_n(\Pi X_j,\star)$, entonces:
    \[
      \Phi[\widetilde{\alpha}]=
      \big\{(q_i\circ\Phi)[\widetilde{\alpha}] \big\}_{i\in J}=
      \big\{(p_i)_{\#}[\widetilde{\alpha}]\big\}_{i\in J}=
      \big\{[p_i\circ \widetilde{\alpha}]\big\}_{i\in J}=
      \big\{[\alpha_i]\big\}_{i\in J}=\alpha
    \]
    Por lo tanto $[\widetilde{\alpha}]$ est\'a en la preimagen de $\alpha$ y as\'i $\Phi$ es
    sobre.
  \item[($\Phi$ es inyectiva)] Sea $[\beta]\in \pi_n(\Pi\,X_j,\star)$ tal que $\Phi[\beta]=1$,
    es decir, cada coordenada de $\Phi[\beta]$ es la clase del lazo constante $e_i:\Sn^n\ra X_i$.
    Entonces:
    \[
      \big\{[e_i]\}_{i\in J}=
      \Phi[\beta]=
      \big\{ [p_i\circ\beta] \big\}_{i\in J}
    \]
    o equivalentemente $e_i\simeq p_i\circ\beta$ para toda $i\in J$.

    Esto quiere decir que existe una familia de homotop\'ias $H_i:\Sn^n\times I \ra X_i$ tales que
    $H_i(s,0)=(p_i\circ\beta)(s)$ y $H_i(s,1)=e_i(s)=x_i$. Adem\'as, como cada $p_i\circ\beta$ es
    basada en el punto $\bullet=(1,0,\ldots,1)\in\Sn^n$, entonces cada homotop\'ia es basada:
    $H_i(\bullet,t)=x_i$ para toda $t\in I$.

    Por lo tanto $\{H_i\}_{i\in J}$ es una familia de morfismos en $\cat{Top}_*$. Por la propiedad
    universal del producto, existe un \'unico morfismo $H:\Sn^n\times I\ra \Pi X_j$ tal que hace
    conmutar el siguiente diagrama:
    \[
      \begin{tikzcd}
        \Sn^n\times I \arrow[rr,dashed,"H"] \arrow[dr,"H_i"'] &&
        \prod X_j \arrow[dl,"p_i"] \\
        & X_i  &
      \end{tikzcd}
    \]

    Observa que
    \begin{align*}
      H_0(s)&=
      \big\{ p_i(H(s,0)) \big\}_{i\in J}=
      \big\{ H_i(s,0) \big\}_{i\in J}=
      \big\{ (p_i\circ \beta )(s) \big\}_{i\in J}=
      \beta(s)\\
      H_1(s)&=
      \big\{ p_i(H(s,1)) \big\}_{i\in J}=
      \big\{ H_i(s,1) \big\}_{i\in J}=
      \big\{ e_i(s) \big\}_{i\in J}=
      e_{\star}(s)
    \end{align*}
    donde $e_{\star}:\Sn^n\ra\Pi\, X_j$ es el lazo constante. Entonces $H$ es una homotop\'ia
    entre las funciones $\beta$ y $e_{\star}$. Por lo tanto $[\beta]=[e_{\star}]=1\in\pi_n(\Pi\, X_j)$
    y $\Phi$ es inyectiva.
  \end{enumerate}
\end{proof}
\end{document}