\documentclass[../../topologia_algebraica]{subfiles}
\begin{document}
\section{El teorema de escisi\'on}

La idea general del teorema de escis\'on es que la homolog\'ia de $X$ relativo a
$A$ es lo mismo que si le quitamos a $X$ un subconjunto $U\subset A$ y calculamos la
homolog\'ia de $X-U$ relativa a $A-U$. En palabras, la homolog\'ia relativa a un
subespacio no nota la diferencia si el quitas (ciertos) conjuntos contenidos en el
subespacio.

\begin{thm}\label{thm:escision}
  Sea $X$ un espacio y $U\subset A\subseteq X$ tal que $\bar{U}\subset\mathring{A}$.
  Entonces la inclusi\'on $j:(X-U,A-U)\hookrightarrow (X,A)$ induce un isomorfismo
  en homolog\'ias, es decir:
  \[
    H_n(j):H_n(X-U,A-U) \stackrel{\cong}{\lra} H_n(X,A)
  \]
  es un isomorfismo.
\end{thm}

En este caso decimos que $U$ se puede \emph{escindir} de la pareja $(X,A)$.

La idea intuitiva es que si tenemos una cadena en $X$ le podemos cortar los
pedazos que est\'an en $U$ de tal manera que la homolog\'ia relativa a $A$
no se de cuenta.

Por ejemplo, supongamos que tenemos una cadena $\tau=\sigma_1+\sigma_2+\sigma_3$
como en el siguiente dibujo

\begin{figure}[h!]%%%%%%%%%%%%%%%%%%%%%%%%%%%%%%%%%%%%%%%%%%%%%%%%%%%%%%%%%%%%%% FIGURE
  \centering
%  \includegraphics[scale=0.11]{dibujo_escision}
\end{figure}%%%%%%%%%%%%%%%%%%%%%%%%%%%%%%%%%%%%%%%%%%%%%%%%%%%%%%%%%%%%%%%%%%%%%

Observa que la cadena $\tau'=\sigma''_1+\sigma_2+\sigma_3\in S_n(X-U,A-U)$, obtenida
``cortando'' la cadena $\tau\in S_n(X,A)$ en el punto $u\in\im{\sigma_1}$, es una
preimagen de $[\tau]\in H_n(X,A)$ bajo $H_n(j)$. En efecto si identifico
$[\tau']=[j_{\#}(\tau')]=H_n(j)[\tau']$ entonces
\[
  \tau-\tau'=\sigma_1-\sigma'_1-\sigma''_1=\partial(\hat{\sigma})-\sigma'_1\in B_n(X,A)
  \quad\then\quad [\tau]=[\tau']\in H_n(X,A)
\]
La idea del teorema de escisi\'on es que esta elecci\'on de preimagen es \'unica
m\'odulo clases de homolog\'ia.

En lugar de probar el teorema de escisi\'on commo viene enunciado en el teorema
\ref{thm:escision}, vamos a probar un teorema m\'as general; a ambas las llamaremos
igual.

Primero definimos un subcomplejo del complejo singular $S_{\bullet}(X)$. Sea
$\fU=\{U_j\}_{j\in J}$ una familia de subespacios tales que $X=\cup_j \mathring{U}_j$
(por ejemplo $\fU$ puedes ser una cubierta abierta). Definimos el complejo de
cadenas $S_{\bullet}^{\fU}(X)$ de la siguiente manera:
\[
  S_n^{\fU}(X)=\Big\{\tau=\sum_{\sigma\in\sS_n(X)}r_{\sigma}\sigma\in S_n(X)\mid
  \forall\sigma,\;\exists U_j\in\fU\;\text{tal que}\;\im{\sigma}\subseteq U_j\Big\}.
\]

Por ejemplo:

\begin{figure}[h!]%%%%%%%%%%%%%%%%%%%%%%%%%%%%%%%%%%%%%%%%%%%%%%%%%%%%%%%%%%%%%% FIGURE
  \centering
%  \includegraphics[scale=0.11]{cadenas_en_cubierta}
\end{figure}%%%%%%%%%%%%%%%%%%%%%%%%%%%%%%%%%%%%%%%%%%%%%%%%%%%%%%%%%%%%%%%%%%%%%

Claramente es un $R$-subm\'odulo de $S_n(X)$, entonces la inclus\'ion
$\imath_{\bullet}:S_{\bullet}^{\fU}\hookrightarrow S_{\bullet}(X)$ es un morfismo de complejos
de cadena. Estamos listos para enunciar el teorema de escisi\'on general:

\begin{thm}\label{thm:escision_general}
  La inclusi\'on $\imath_{\bullet}^{\fU}:S_{\bullet}^{\fU}(X)\hookrightarrow S_{\bullet}(X)$
  induce un ismorfismo en homolog\'ias, es decir:
  \[
    H_n(S_{\bullet}^{\fU}(X))\cong H_n(S_n(X)) \stackrel{\text{def}}{=} H_n(X).
  \]
\end{thm}

Para probar este teorema, nos vamos a basar en una propiedad importante de los complejos
de cadena:

\import{\directory}{ejercicios/71} %%%%%%%%%%%%%%%%%%%%%%%%%%%%%%%%%%%%%%%%%%%%%% EJERCICIO 71

Gracias a este ejercicio, simplemente hay que probar que la inclus\'on
$\imath:=\imath_{\bullet}^{\fU}$ es una equivalencia homot\'opica de cadenas, es decir
necesitamos encontrar un inverso homot\'opico de $\imath$. M\'as precisamente, necesitamos
definir un morfismo $S_{\bullet}\ra S_{\bullet}^{\fU}$ (ie. una transformaci\'on natural
entre los funtores $S_{\bullet}(\_)$ y $S_{\bullet}^{\fU}(\_)$) que compuesto con $\imath$
sea homot\'opico a la identidad.

Antes que eso vamos a contruir el morfismo de complejos de cadena que simplemente es
hacer subdivisi\'on baric\'entrica; una aplicaci\'on repetida de este morfismo nos
llevar\'a a definir el morfismo $S_{\bullet}\ra S_{\bullet}^{\fU}$ que buscamos.

Para definir la subdivisi\'on baric\'entrica como un morfismo de complejos de cadena,
vamos a seguir un m\'etodo muy similar a la prueba de la invariancia homot\'opica de
la homolog\'ia singular (en particular en la construcci\'on de la homotop\'ia
$\la^1_{\#}\simeq\la^0_{\#}$).

Definimos $\text{sd}_{\bullet}^X:S_{\bullet}(X)\ra S_{\bullet}(X)$ de la siguiente manera:
queremos que sea un morfismo de cadenas entonces en particular queremos que
\begin{equation}\label{eq:diagramas_subdivision_baricentrica}
  \begin{tikzcd}[column sep=large]
    S_n(\Delta^n) \arrow[r,"\text{sd}_n^{\Delta^n}"] \arrow[d,"\sigma_{\#}"'] &
    S_n(\Delta^n) \arrow[d,"(F^i_n)_{\#}"] \\
    S_n(X) \arrow[r,"\text{sd}_n^X"] & S_n(X)
  \end{tikzcd}\qquad
  \begin{tikzcd}[column sep=large]
    S_n(\Delta^{n-1}) \arrow[r,"\text{sd}_{n-1}^{\Delta^{n-1}}"] \arrow[d,"(F^i_n)_{\#}"'] &
    S_n(\Delta^{n-1}) \arrow[d,"\sigma_{\#}"] \\
    S_n(\Delta^n) \arrow[r,"\text{sd}_n^{\Delta^n}"] & S_n(\Delta^n)
  \end{tikzcd}
\end{equation}
sean un diagraman conmutativon, entonces para definir sd${}_n^X(\sigma)$ basta definir
$\sQ_n:\text{sd}_n^{\Delta^n}(\Id_{\delta^n})$ porque:
\[
  \text{sd}_n^X(\sigma)=\text{sd}_n^{\Delta^n}(\sigma_{\#}(\Id_{\Delta^n}))=
  \sigma_{\#}\big(\text{sd}_n^{\Delta^n}(\Id_{\Delta^n}) \big)=\sigma_{\#}(\sQ_n)
\]

Como $\sQ_n$ es hacer la subdivisi\'on baric\'entrica normal de $\Delta^n$, entonces
es simplemente agregar el baricentro $\fb_n:=\fb(\Delta^n)$ como v\'ertice a la frontera
de $\Delta^n$, o m\'as precisamente, a la frontera de la imagen de $\Id_{\Delta^n}$. Es
decir, queremos definir $\sQ_n=\fb_n\cdot\text{sd}_{n-1}^{\Delta^n}(\partial_n(\Id_{\Delta^n}))$.

Podemos definir $\sQ_n$ directamente por inducci\'on para forzar las propiedades que queremos
que tenga sd${}_n^X$. Para $n=0$, tenemos que $\sQ_0\in S_0(X)$ entonces necesariamente
$\sQ_0=\Id_{\Delta^0}$ porque est\'a es la \'unica funci\'on $\Delta^0\ra\Delta^0$ que existe.
En general tenemos:
\begin{align*}
  \text{sd}_{n-1}^{\Delta^n}(\partial_n(\Id_{\Delta^n})) & =
  \text{sd}_{n-1}^{\Delta^n}\paren{\sum_{i=0}^n(-1)^i F_n^i}=
  \sum_{i=0}^n(-1)^i(F^i_n)_{\#}
  \Big(\underset{\sQ_{n-1}}{\underbrace{\text{sd}_{n-1}^{\Delta^{n-1}}(\Id_{\Delta^{n-1}})}}\Big) \\ & =
  \sum_{i=0}^n(-1)^i(F^i_n)_{\#}(\sQ_{n-1})
\end{align*}
Por lo tanto podemos definir
\[
  \sQ_n:=\fb_n\cdot\underset{\fz_{n-1}}%
    {\underbrace{\paren{\sum_{i=0}^n(-1)^i(F^i_n)_{\#}(\sQ_{n-1})}}}
\]
Esta definici\'on cumple una propiedad importante (casi id\'entico al lema
\ref{lema:calculo_frontera_sP}):

\begin{lema}\label{lema:calculo_frontera_sQ}
  $\partial_n(\sQ_n)=\fz_{n-1}$ para $n>0$.
\end{lema}
\begin{proof}
  Claramente $\partial_0(\sQ_0)=0$ y tambi\'en
  \[
    \partial_1(\sQ_1)=\partial_1\paren{\fb_1\cdot (F^0_1)_{\#}(\sQ_0)-\fb_1\cdot(F^0_1)_{\#}(\sQ_0)}=
    \partial_1(\fb_1\cdot e_1 -\fb_1\cdot e_0)=e_1-e_0=\sum_{i=0}^1(-1)^i(F^i_1)_{\#}(\sQ_0)=\fz_1.
  \]
  Ahora por inducci\'on, supongamos que $\partial_n(\sQ_n)=\fz_{n-1}$. Observa que
  $\partial_{n-1}(\fz_{n-1})=\partial_{n-1}(\partial_n(\sQ_n))=0$, entonces:
  \begin{align*}
  \partial_n(\fz_n) & =\partial_n\paren{\sum_{i=0}^{n+1}(-1)^i(F^i_{n+1})_{\#}(\sQ_{n})}=
  \sum_{i=0}^{n+1}(-1)^i(F^i_{n+1})_{\#}(\partial_n(\sQ_{n}))\\ & =
  \sum_{i=0}^{n+1}(-1)^i(F^i_{n+1})_{\#}(\fz_{n-1})=
  \sum_{i=0}^{n+1}(-1)^i(F^i_{n+1})_{\#}\paren{\sum_{j=0}^n(-1)^j(F^j_n)_{\#}(\sQ_{n-1})} \\ & =
  \sum_i \sum_j (-1)^{i+j}(F^i_{n+1}\circ F^j_n)_{\#}(\sQ_{n-1})=0
  \end{align*}
  donde la \'ultima igualdad se cumple por el ejercicio \ref{ej:61} (cf. el lema
  \ref{lema:calculo_frontera_sP}). Ahora, con el lema \ref{lema:dos_invariancia}, podemos
  concluir que:
  \[
    \partial_{n+1}(\sQ_{n+1})=
    \partial_{n+1}(\fb_{n+1}\cdot\fz_n)=
    \fz_n-\fb_{n+1}\cdot\cancelto{0}{\partial_n(\fz_n)}\;\;=
    \fz_n.
  \]
  
\end{proof}

Con esto ya podemos definir la subdivisi\'on baric\'entrica a nivel de complejos de
cadena:

\begin{defin}
  La subdivisi\'on baric\'entrica $\text{sd}_{\bullet}^X:S_{\bullet}(X)\ra S_{\bullet}^X$ de
  complejos de cadena se define como
  \[
    \text{sd}^X_n:S_n(X)\lra S_n(X) \quad\text{con}\quad \sigma\mapsto \sigma_{\#}(\sQ_n).
  \]
\end{defin}

\import{\directory}{ejercicios/72} %%%%%%%%%%%%%%%%%%%%%%%%%%%%%%%%%%%%%%%%%%%%%% EJERCICIO 72

\begin{lema}\label{lema:sd_homotopico_identidad}
  sd${}^X_{\bullet}:S_{\bullet}(X)\ra S_{\bullet}(X)$ es un morfismo de complejos de cadenas y
  sd${}^X_{\bullet}\simeq\Id_{S_{\bullet}(X)}$
\end{lema}
\begin{proof}
  Primero probamos que es un morfismo de complejos de cadenas, es decir que para toda $n$
  el diagrama
  \[
    \sD_n:=\quad
    \begin{tikzcd}[column sep=large]
      S_n(X) \arrow[r,"\partial_n"] \arrow[d,"\text{sd}^X_n"] & S_{n-1}(X) \arrow[d,"\text{sd}^X_{n-1}"]\\
      S_n(X) \arrow[r,"\partial_n"] & S_{n-1}(X)
    \end{tikzcd}
  \]
  es conmutativo; esto lo hacemos por inducci\'on. El caso $n=0$ es trivial porque
  $\partial_0=0$. Para el caso general, sea $\sigma\in S_n(X)$, entonces
  \[
    \partial_n(\text{sd}^X_n(\sigma))=
    \partial_n(\sigma_{\#}(\sQ_n))=
    \partial_n\big(\sigma_{\#}(\fb_n\cdot\text{sd}^{\Delta^n}_{n-1}(\partial_n(\Id_{\Delta^n})))\big)=
    \sigma_{\#}\Big( \partial_n(\fb_n\cdot\text{sd}^{\Delta^n}_{n-1}(\partial_n(\Id_{\Delta^n}))) \Big)
  \]
  que por el lema \ref{lema:dos_invariancia} se hace:
  \begin{align*}
    \partial_n(\text{sd}^X_n(\sigma))&=
    \sigma_{\#}\big( \text{sd}^{\Delta^n}_{n-1}(\partial_n(\Id_{\Delta^n})) \big)-
    \fb_n\cdot\partial_{n-1}\text{sd}^{\Delta^n}_{n-1}(\partial_n(\Id_{\Delta^n})) \\ & =
    \sigma_{\#}\big( \text{sd}^{\Delta^n}_{n-1}(\partial_n(\Id_{\Delta^n})) \big)-
    \fb_n\cdot\text{sd}^{\Delta^n}_{n-1}(\cancelto{0}{\partial_{n-1}\partial_n(\Id_{\Delta^n})})\\ & =
     \sigma_{\#}\big( \text{sd}^{\Delta^n}_{n-1}(\partial_n(\Id_{\Delta^n})) \big).
  \end{align*}
  donde el intercambio de orden de $\partial_{n-1}$ y $\text{sd}^{\Delta^n}_{n-1}$ se
  vale por la hip\'otesis de inducci\'on.
  
  Como $\text{sd}^{X}_{\bullet}$ es una transformaci\'on natural (ejercicio \ref{ej:72}),
  entonces podemos intercambiar $\sigma_{\#}$ con $\text{sd}^{\Delta^n}_{n-1}$ para concluir:
  \[
    \partial_n(\text{sd}^{X}_{n}(\sigma))=
    \text{sd}^{X}_{n-1}(\sigma_{\#}(\partial_n(\Id_{\Delta^n})))=
    \text{sd}^X_{n-1}(\partial_n(\sigma_{\#}(\Id_{\delta^n})))=
    \text{sd}^X_{n-1}(\partial_n(\sigma))
  \]
  y el diagrama $\sD_n$ es conmutativo.

  En lo que sigue omitir\'e muchos detalles porque son exactamente los mismos
  detalles, y pasos, que usamos en la prueba de la invariancia homot\'opica:
  simplemente hay que cambiar $\la^1_{\#}$ por $\text{sd}^X_{\bullet}$ y $\la^0_{\#}$
  por $\Ii:=\Id_{S_{\bullet}(X)}$. Incluso usar\'e la misma notaci\'on.
  
  Ahora necesitamos una familia de morfismos $\fT=\{T^X_n:S_n(X)\ra S_{n+1}(X)\}$ que
  que sea la homotop\'ia entre sd$^X_{\bullet}$ y $\Ii$, es decir
  que $\partial_{n+1}T^X_n+T^X_{n+1}\partial_n=\text{sd}^X_n-\Ii$. Como queremos que
  $T^X_n$ sea una transformaci\'on natural entre los funtores $S_n(\_)$ y $S_{n+1}(\_)$,
  entonces basta definir:
  \[
    T^X_n(\sigma):=
    \sigma_{\#}\big(\underset{\sT_n}{\underbrace{T^{\Delta^n}_n(\Id_{\Delta^n})}} \big)=
    \sigma_{\#}(\sT_n)
  \]
  donde las $\sT_n$ cumplen
  \begin{equation}\label{eq:subdivision_homotopico_identidad}
    \partial_{n+1}(\sT_n)+\sum_{i=0}^n(-1)^i(F^i_n)_{\#}(\sT_{n-1})=\sQ_n-\Id_{\Delta^n}
  \end{equation}
  porque est\'an definidas recursivamente como
  \[
    \sT_0=\Id_{\Delta^0}
  \]
  \[
    \sT_n=\fb_n\cdot\Big(\underset{\fz_n}{\sQ_n-\Id_{\Delta^n}-
      \underbrace{\sum_{i=0}^n(-1)^i(F^i_n)_{\#}(\sQ_{n-1})}} \Big).
  \]
  que cumplen que $\partial_{n+1}(\sT_n)=\fz_n$.

  Gracias a esta propiedad la ecuaci\'on (\ref{eq:subdivision_homotopico_identidad})
  se cumple y as\'i $\fT$ es una homotop\'ia $\text{sd}^X_{\bullet}\sim\Ii$.
\end{proof}

Este lema es valioso, b\'asicamente nos dice que podemos iterar la subdivisi\'on
baric\'entrica sobre un complejo de cadenas sin alterar las homomolog\'ia, en efecto:
si sd$^X_{\bullet}\simeq\Id$ entonces $H_n(\text{sd}^X_{\bullet})=H_n(\Id)=\Id$. Esto
sugiere una manera de contruir una funci\'on $S_{\bullet}(X)\ra S_{\bullet}^{\fU}$
que sea el inverso homot\'opico de la inclusi\'on $\imath: S_{\bullet}^{\fU}(X)\ra S_{\bullet}(X)$:
simplemente sigue subdividiendo una $n$-cadena hasta que todas sus caras sean
suficientemente peque\~nas para que lo que quede sea un elemento de $S_n^{\fU}(X)$.

Para formalizar esta noci\'on de hacerse m\'as peque\~no, definimos:

\begin{defin}
  Sea $A$ un subespacio de un espacio m\'etrico $(X,d)$. El \emph{di\'ametro} de $A$ se
  define como
  \[
    \text{diam}(A):=\sup\{d(x,y)\mid x,y\in X\}.
  \]
  Adem\'as, si $X$ es compacto y $\fU:=\{U_{\la}\}_{\la\in\Lambda}$ es una cubierta abierta
  de $X$, existe una $\eps>0$ tal que si diam$(A)<\eps$, entonces $A\subseteq U_{\la}$
  para alguna $\la\in\Lambda$; a esta $\eps$ se le llama el \emph{n\'umero de Lebesgue}
  de $\fU$.
\end{defin}

Si $\sigma:\Delta^n\ra\RR^N$ es un simplejo af\'in, entonces su subdivisi\'on baric\'entrica
contiene subsimplejos de di\'ametro menor o igual a $(\tfrac{n}{n+1})$diam$(\sigma[\Delta^n])$.
Por lo tanto si subdividimos $r$ veces el di\'ametro de sus subsimplejos es menor o igual
a $(\tfrac{n}{n+1})^r$diam$(\sigma[\Delta^n])$.

Esto nos permite demostrar rigurosamente nuestra intuici\'on de que existen una cantidad
finita de pasos de subdivisi\'on baric\'entrica que meten a un simplejo a $S_{n}^{\fU}(X)$.

\begin{lema}
  Sea $\fU=\{U_{\la}\}_{\la\in\Lambda}$ una familia de subespacios de $X$ tales que
  $X=\cup \mathring{U}_{\la}$, entonces
  \[
    \forall \sigma\in S_n(X) \quad\exists r\geq0\;\;\text{tal que}\;\;
    \big(\text{sd}^X_n\big)^r(\Delta^n)\in S_n^{\fU}(X).
  \]
\end{lema}
\begin{proof}
  Observa que la cubierta abierta $\{\mathring{U}_{\la}\}_{\la}$ induce la cubierta abierta
  $\sigma^{-1}(\fU)=\{\sigma^{-1}[\mathring{U}_{\la}]\}$ de $\Delta^n$. Como \'este es
  compacto, la cubierta $\sigma^{-1}[\fU]$ tiene un n\'umero de Lebesgue $\eps>0$, como
  $(\tfrac{n}{n+1})^r\ra0$ cuando $r\rai$, y diam$(\Delta^n)$ es constante, entonces
  existe una $r$ suficientemente grande tal que
  \[
    \text{diam}\big( \big(\text{sd}^{\RR^N}_n\big)^r(\Delta^n)\big)=
    \paren{\tfrac{n}{n+1}}^r\text{diam}(\sigma[\Delta^n])<\eps.
  \]
  Esto quiere decir que los subcomplejos de $\big(\text{sd}^{\RR^N}_n\big)^r(\Delta^n)$
  tienen di\'ametro menor que $\eps$.

  Por lo tanto si $\tau$ es un subsimplejo de
  $(\text{sd}^X_n)^r(\sigma)=\sigma_{\#}\big(\text{sd}^{\RR^N}_n\big)^r(\Delta^n)$, entonces
  $\tau=\sigma\circ\tau'$ donde $\tau'$ es un subsimplejo de
  $\big(\text{sd}^{\RR^N}_n\big)^r(\Delta^n)$, es decir que diam$(\im{\tau'})<\eps$ que
  implica que para tod sub simplejo  $\tau$ hay un $U_{\la}$ tal que
  \[
    \tau'[\Delta^n]\subseteq\sigma^{-1}[\mathring{U}_{\la}] \quad\then\quad
    (\sigma\circ\tau')[\Delta^n]=\tau[\Delta^n]\subseteq \mathring{U}_{\la}\subset U_{\la} \quad\then\quad
    (\text{sd}^X_n)^r(\sigma)\in S_n^{\fU}(X).
  \]
\end{proof}

Este lema sugiere definir el m\'inimo n\'umero de subdivisiones necesarias para que un
simplejo sea elemento de $S_n^{\fU}(X)$:
\[
  \fr:S_n(X)\ra\NN \quad\text{con}\quad
  \sigma\mapsto\min\Big\{r\in\NN\mid (\text{sd}^X_n)^r(\sigma)\in S_n^{\fU}(X)\Big\}
\]

Lo ideal ser\'ia definir $\sigma\mapsto (\text{sd}^X_n)^{\fr(\sigma)}(\sigma)$ como
el morfismo de cadenas $S_{\bullet}(X)\ra S_{\bullet}^{\fU}(X)$ que buscamos, pero no
se puede hacer tan directamente:

\import{\directory}{ejercicios/73} %%%%%%%%%%%%%%%%%%%%%%%%%%%%%%%%%%%%%%%%%%%%%% EJERCICIO 73

Esto se puede arreglar. Consideramos la homotop\'ia $\fT=\{T^X_n:S_n(X)\ra S_{n+1}(X)\}$
de la prueba del lema \ref{lema:sd_homotopico_identidad} y definimos:
\[
  R_n:S_n(X)\ra S_n(X) \quad\text{con}\quad
  R_n(\sigma):=\sum_{i=0}^{\fr(\sigma)-1}T_n^X\big((\text{sd}^X_n)^i(\sigma)\big).
\]
Observa que
\[
  R_{n-1}(\partial_n(\sigma))=
  R_{n-1}\paren{\sum_{i=0}^n(-1)^i\sigma^{(i)}}=
  \sum_{i=0}^n(-1)^iR_{n-1}(\sigma^{(i)})=
  \sum_{i=0}^n(-1)^i\sum_{j=0}^{\fr(\sigma^{(i)})-1}T_{n-1}^X\big((\text{sd}^X_{n-1})^j(\sigma^{(i)})\big).
\]
Por otro lado:
\[
  \partial_{n+1}(R_n(\sigma))=
  \sum_{i=0}^{\fr(\sigma)-1}(\partial_{n+1}T_n^X)\big((\text{sd}^X_n)^i(\sigma)\big).
\]
Recuerda que los morfismos $T^X_n$ cumplen $\partial_{n+1}T^X_n+T^X_{n-1}\partial_n=\text{sd}^X_n-\Id$,
o equivalentemente $\partial_{n+1}T^X_n=\text{sd}^X_n-\Id-T^X_{n-1}\partial_n$. Si sustituimos
esto en la f\'ormula anterior, obtenemos:
\begin{align*}
  \partial_{n+1}(R_n(\sigma)) & =
  \sum_{i=0}^{\fr(\sigma)-1}%
  (\text{sd}^X_n-\Id-T^X_{n-1}\partial_n)\big((\text{sd}^X_n)^i(\sigma)\big)\\ & =
  \sum_{i=1}^{\fr(\sigma)}(\text{sd}^X_n)^i(\sigma)-\sum_{i=0}^{\fr(\sigma)-1}(\text{sd}^X_n)^i(\sigma)-
  \sum_{i=0}^{\fr(\sigma)-1}T^X_{n-1}(\partial_n(\text{sd}^X_n)^i)(\sigma)\\ & =
  (\text{sd}^X_n)^{\fr(\sigma)}(\sigma)-\sigma-
  \sum_{i=0}^{\fr(\sigma)-1}T^X_{n-1}(\partial_n(\text{sd}^X_n)^i)(\sigma)
\end{align*}
donde la \'ultima suma es:
\[
  \sum_{i=0}^{\fr(\sigma)-1}T^X_{n-1}(\partial_n(\text{sd}^X_n)^i)(\sigma)=
  \sum_{i=0}^{\fr(\sigma)-1}T^X_{n-1}((\text{sd}^X_{n-1})^i\partial_n)(\sigma)=
  \sum_{i=0}^{\fr(\sigma)-1}\sum_{j=0}^n(-1)^j T^X_{n-1}\big((\text{sd}^X_{n-1})^i(\sigma^{(i)})\big)
\]
porque $\text{sd}^X_n$ conmuta con $\partial_n$.

Por lo tanto:
\begin{equation}\label{eq:homotopia_final}
  (\partial_{n+1}R_n+R_{n-1}\partial_n)(\sigma)=
  \sigma-\underset{t_n(\sigma)}{\underbrace{\left[(\text{sd}^X_n)^{\fr(\sigma)}(\sigma)+
    \sum_{j=0}^n(-1)^j\sum_{i=\fr(\sigma^{(i)})}^{\fr(\sigma)}%
    T^X_{n-1}\big((\text{sd}^X_{n-1})^i(\sigma^{(i)})\big)\right]}}.
\end{equation}
Entonces si redefinimos $t_{\bullet}$ como lo que est\'a en corchetes, tendremos que
$\Id\simeq t_{\bullet}$ mediante la familia $\{R_n\}_{n\in\ZZ}$ de morfismos. Entonces
si definimos:
\[
  t_n:S_n(X)\ra S_n(X)\quad\text{con}\quad
  t_n(\sigma)=\sigma-(\partial_{n+1}R_n+R_{n-1}\partial_n)(\sigma)
\]
estamos en posici\'on para probar el teorema de escisi\'on general:

\begin{proof}(teorema \ref{thm:escision_general})
Observa que $t_{\bullet}=(t_n)$ es un morfismo de complejos de cadena porque
\[
  (\partial_nt_n)(\sigma)=\partial_n(\sigma-(\partial_{n+1}R_n+R_{n-1}\partial_n)(\sigma))=
  \partial_n(\sigma)+\partial_nR_{n-1}\partial_n(\sigma),
\]
y porque
\[
  t_{n-1}\partial_n(\sigma)=\partial_n(\sigma)+\partial_{n}R_{n-1}\partial_n(\sigma)+
  \cancelto{0}{R_{n-2}\partial_{n-1}\partial_n(\sigma)}.
\]

Ahora, si $\im{t_{\bullet}}\subset S_n^{\fU}$, entonces $t_{\bullet}$ se factoriza
a trav\'es de la inclusi\'on $S_{\bullet}^{\fU}(X)\hookrightarrow S_{\bullet}(X)$.
Por lo tanto $\Id\simeq t_{\bullet}=\imath\circ t_{\bullet}$. Para ver la expresi\'on
inversa, sabemos que
\[
  \sigma\in S_{n-1}^{\fU}(X) \quad\then\quad \fr(\sigma)=0 \quad\then\quad R_n(\sigma)=0
\]
porque la suma que define $R_n(\sigma)$ es vac\'ia. Una consecuencia directa de esto
es que
\[
  \sigma\in S_n^{\fU}(X)\quad\then\quad
  \partial_{n}(\sigma)\in S_{n-1}^{\fU}(X)\quad\then\quad
  R_{n-1}(\partial_n(\sigma))=0
\]
Por lo tanto
\[
  t_{n}(\sigma)=\sigma+\partial_{n+1}R_n(\sigma)+R_{n-1}\partial_n(\sigma)=\sigma \quad\then\quad
  \Id=t_{\bullet}\circ\imath=t_{\bullet}|_{S_{\bullet}^{\fU}} \quad\then\quad
  \Id\sim\imath\circ t_{\bullet}.
\]
y podemos conlcuir que $\Id\simeq \imath \circ t_{\bullet}$. Esto quiere decir que
$\imath$ es una equivalencia homot\'opica y por el ejercicio \ref{ej:71} $H_n(\imath)$
es un isomorfismo.

En este \'ultimo p\'arrafo probamos que si $\im{t_{\bullet}}\subseteq S_n^{\fU}(X)$ entonces
$H_n(\imath)$ es un isomorfismo como queremos. El primer t\'ermino de la definici\'on de
$t_n(\sigma)$ claramente cumple $(\text{sd}^X_n)^{\fr(\sigma)}(\sigma)\in S_n^{\fU}(X)$. El
segundo t\'ermino es una suma de elementos de la forma $(\text{sd}_{n-1}^X)^j(\sigma^{(i)})$
donde $\fr(\sigma^{(i)})\leq j$, ie. $(\text{sd}_{n-1}^X)^j(\sigma^{(i)})$. Ahora sea
$\sigma\in S_n^{\fU}(X)$ arbitrario, entonces por definici\'on $T^X_n(\sigma)=\sigma_{\#}(\sT_n)$
pero como $\im{\sigma}\subseteq U_{\la}$ entonces $\im{\sigma_{\#}(\sT_n)}\subseteq U_{\la}$
y as\'i $T^X_n(\sigma)\in S_n^{\fU}(X)$. Si juntamos los dos enunciados cuncluimos que
$t_n(\sigma)$ es una suma de elmentos de $S_n^{\fU}(X)$. Deducimos que $t_n(\sigma)\in S_n^{\fU}(X)$
para toda $\sigma\in S_n(X)$ y acabamos.
\end{proof}

Por \'ultimo vamos a probar ell caso particular del teorema de escisi\'on:

\begin{proof}(teorema \ref{thm:escision})
  Definimos $\fU:=\{X-U,A\}$ como la cubierta. Esta elecci\'on es v\'alida porque
  $\bar{U}\subset\mathring{A}$ implica que
  $X=\mathring{(X-U)}\cup \mathring{A}=(X-\bar{U})\cup\mathring{A}$.

  La inclusi\'on $j:(X-U,A-U)\hookrightarrow(X,A)$ induce un morfismo
  $j_{\#}:S_{\bullet}(X-U)\ra S_{\bullet}(X)$ que cumple que
  \[
    j_{\#}[S_{\bullet}(A-U)]\subseteq S_{\bullet}(A).
  \]
  Por lo tanto $j_{\#}$ pasa al cociente:
  \[
    \overline{j_{\#}}:\frac{S_{\bullet}(X-U)}{S_{\bullet}(A-U)}\lra\frac{S_{\bullet}(X)}{S_{\bullet}(A)}.
  \]
  Hay que probar que $\overline{j_{\#}}$ es un isomorfismo.
  
  Observa que
  \[
    S_n^{\fU}(X)=S_n(X-U)+S_n(A)\subset S_n(X) \quad\text{y}\quad
    S_n(X-U)\cap S_n(A)=S_n((X-U)\cap A)=S_n(A-U).
  \]
  El segundo teorema de isomorfismo dice que la inclusi\'on
  $i:S_{\bullet}(X-U)\hookrightarrow S_{\bullet}(X-U)+S_{\bullet}(A)=S_n^{\fU}(X)$
  induce un isomorfismo:
  \[
    \bar{i}:\frac{S_{\bullet}(X-U)}{S_{\bullet}(A-U)}\stackrel{\cong}{\lra}
    \frac{S_{\bullet}^{\fU}(X)}{S_{\bullet}(A)}.
  \]
  Por otro lado, la inclusi\'on $\imath:S_{\bullet}^{\fU}(X)\ra S_{\bullet}(X)$
  induce un morfismo:
  \[
    \bar{\imath}:\frac{S_{\bullet}^{\fU}(X)}{S_{\bullet}(A)} \lra \frac{S_n(X)}{S_n(A)}
  \]
  porque $\imath[S_{\bullet}(A)]\subseteq S_{\bullet}(A)$. Observa que
  \[
    \begin{tikzcd}
      \tfrac{S_{\bullet}(X-U)}{S_{\bullet}(A-U)} \arrow[rr,"\overline{j_{\#}}"]
      \arrow[dr,"\bar{i}"'] && \tfrac{S_{\bullet}(X)}{S_{\bullet}(A)} \\
      & \tfrac{S_{\bullet}^{\fU}(X)}{S_{\bullet}(A)} \arrow[ur,"\bar{\imath}"'] &
    \end{tikzcd}\quad \overline{j_{\#}}=\bar{\imath}\circ\bar{i}
  \]
  es un diagrama conmutativo porque $j_{\#}=\imath\circ i$. Tomando homolog\'ias, obtenemos:
  \[
    \begin{tikzcd}[row sep=large]
      H_n(X-U,A-U) \arrow[rr,"H_n(\overline{j_{\#}})"] \arrow[dr,"H_n(\bar{i})"'] && H_n(X,A) \\
      & H_n\paren{\tfrac{S_{\bullet}^{\fU}(X)}{S_{\bullet}(A)}} \arrow[ur,"H_n(\bar{\imath})"'] &
    \end{tikzcd}
  \]
  Como $\bar{\imath}$ es un isomorfismo de complejos de cadena, entonces $H_n(\bar{i})$
  es un isomorfismo. Por lo tanto si probamos que $H_n(\bar{\imath})$ es un isomorfismo
  pdor\'iamos conlcuir que $H_n(\overline{j_{\#}})$ es un isomorfismo y acabar\'iamos.
  
  Resulta que $\bar{\imath}$ es una equivalencia homot\'opica de complejos de cadena con
  inverso $\bar{t_{\bullet}}$, el inducido por $t_{\bullet}:S_{\bullet}(X)\ra S_{\bullet}^{\fU}(X)$.
  Si este es el caso, entonces $H_n(\bar{\imath})$ es un isomorfismo y acabamos. Por lo tanto
  redujimos el problema a:
  
\import{\directory}{ejercicios/74} %%%%%%%%%%%%%%%%%%%%%%%%%%%%%%%%%%%%%%%%%%%%%% EJERCICIO 74
\end{proof}

Decimos que la inclusi\'on $\imath$ (o $\bar{\imath}$) es un \emph{retracto por deformaci\'on},
es decir: existe un morfismo $r$ (que en este caso es $r=t_{\bullet}$) tal que $\imath\circ r=\Id$
y $r\circ\imath\simeq\Id$.

\end{document}