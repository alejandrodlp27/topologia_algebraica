\documentclass[../../topologia_algebraica]{subfiles}
\begin{document}
\section{Invariancia homot\'opica de la homolog\'ia singular}\label{sec:invariancia}


Primero fijo un anillo $R$ para no tener que escribirlo repetidamente.
El prop\'osito de esta secci\'on es probar que si dos espacios son homot\'opicos, entonces sus
homolog\'ias coinciden.

\begin{thm}\label{thm:invariancia_homologia}
  $X\simeq Y \quad\then\quad H_n(X)\cong H_n(Y)$
\end{thm}

Claramente esto se sigue de que el funtor $X\ra H_n(X)$ se factoriza a trav\'es de clases de
homotop\'ia, es decir:

\begin{thm}\label{thm:invariancia_homologia_dos}
  Si $f:X\ra Y$ y $g:X\ra Y$ son funciones continuas homot\'opicas, entonces inducen el mismo
  morfismo en homolog\'ia, es decir:
  \[
    f\simeq g \quad\then\quad H_n(f)=H_n(g)\quad\forall n\in\ZZ.
  \]
\end{thm}

Sea $F:X\times I\ra Y$ la homotop\'ia entre $f$ y $g$. Entonces $F_0=f$ y $F_1=g$, o
equivalentemente:
\[
  f=F\circ\la^0 \;\; , \;\; f=F\circ\la^1 \quad\text{donde}\quad
  \la^i:X\hookrightarrow X\times\{i\}\subset X\times I.
\]
Observa que $H_n(f)=H_n(F\circ \la^0)=H_n(F)\circ H_n(\la^0)$, entonces:
\[
  H_n(\la^0)=H_n(\la^1) \quad\then\quad
  H_n(F)\circ H_n(\la^0)=H_n(F)\circ H_n(\la^1) \quad\then\quad
  H_n(f)=H_n(g).
\]
Por lo tanto probar el teorema se reduce a probar que $H_n(\la^0)=H_n(\la^1)$.
Sabemos que $H_n(\la^1)=H_n(\la^0)$ si para toda $[\sigma]\in H_n(X)$ se tiene
que $H_n(\la^0)[\sigma]-H_n(\la^1)[\sigma]=0\in H_n(X\times I)$ o equivalentemente
\[
  0=[\la_{\#}^0(\sigma)]-[\la_{\#}^1(\sigma)]=[\la_{\#}^0(\sigma)-\la_{\#}^1(\sigma)]
  \quad\iff\quad (\la_{\#}^0-\la_{\#}^1)(\sigma)\in B_n(X\times I)\quad\forall\sigma\in S_n(X).
\]
La forma de probar esto es encontrar una familia de morfismos
$\{T_n^X:S_n(X)\ra S_{n+1}(X\times I)\}_{n\in\ZZ}$ que cumplen
\begin{equation}
  \label{eq:homotopia_complejos_lambda}
  \la_{\#}^1-\la_{\#}^0=\partial'_{n+1}\circ T_n+T_{n-1}\circ\partial_n \quad\forall n\in\ZZ
\end{equation}
donde
\[
  \begin{tikzcd}
    \cdots \arrow[r] & S_{n+1}(X) \arrow[r,"\partial_{n+1}"] \arrow[d,"\la^i_{n+1}"'] &
    S_n(X) \arrow[r,"\partial_n"] \arrow[d,"\la^i_n"'] \arrow[ld,dashed,"T^X_n"'] &
    S_{n-1}(X) \arrow[r] \arrow[ld,dashed,"T^X_{n-1}"'] \arrow[d,"\la_{n-1}^i"] & \cdots \\
    \cdots \arrow[r] & S_{n+1}(X\times I) \arrow[r,"\partial'_{n+1}"'] &
    S_n(X\times I) \arrow[r,"\partial'_n"'] & S_{n-1}(X\times I) \arrow[r] & \cdots
  \end{tikzcd}
\]

En general esto se llama una homotop\'ia entre complejos:

\begin{defin}
  Sean $\Cc_{\bullet}=(C_n,\partial_n)$ y $\Dd_{\bullet}=(D_n,\partial_n)$ dos complejos
  de $R$-m\'odulos y sean $f,g:\Cc\ra\Dd$ morfismos de complejos. Una \emph{homotop\'ia
    de complejos de cadena} es una familia de morfismo de $R$-m\'odulos
  $\fT=\{T_n:C_n\ra D_{n+1}\}_{n\in\ZZ}$ que cumplen
  \[
    f_n-g_n=\partial'_{n+1}\circ T_n + T_{n-1}\circ\partial_n\quad\forall n\in\ZZ.
  \]
  Si dos morfismos son homot\'opicos, lo denotamos por $f\simeq g$, y si existe una
  homotop\'ia entre dos complejos de cadena, decimos que son homot\'opicos y lo denotamos
  $\Cc_{\bullet}\simeq\Dd_{\bullet}$.
\end{defin}

\begin{nota}
  Si $f\simeq g$ entonces $H_n(f)=H_n(g)$ porque si $[z]\in H_n(\Cc_{\bullet})$ entonces
  \[
    (H_n(f)-H_n(g))[z]=[(f_n-g_n)(z)]=
    \big[(\partial'_{n+1}\circ T_n)(z) + (T_{n-1}\circ\partial_n)(z)\big]=0.
  \]
  Por lo tanto dos morfismos de complejos de cadena que son homot\'opicos inducen el
  mismo morfismo en homolog\'ias.

  Para probar (\ref{thm:invariancia_homologia_dos}), hay que probar que
  $\la_{\#}^0\simeq\la_{\#}^1$, es decir hay que encontrar una familia
  $\mathfrak{T}=\{T^X_n:S_n(X)\ra S_{n+1}(X\times I)\}$ que cumpla (\ref{eq:homotopia_complejos_lambda}).
\end{nota}

Para definir las $T^X_n$, en general si $f:X\ra Y$ es continua, entonces queremos que
\[
  \begin{tikzcd}[column sep=large]
    S_n(X) \arrow[r,"T^X_n"] \arrow[d,"f_{\#}"'] & S_{n+1}(X\times I) \arrow[d,"(f\times\Id)_{\#}"]\\
    S_n(X) \arrow[r,"T^Y_n"] & S_{n+1}(Y\times I)
  \end{tikzcd}
\]
sea conmutativo. En particular para $\sigma:\Delta^n\ra X$, ie. $\sigma\in S_n(X)$, queremos
que:
\[
 (\sigma\times\Id_I)_{\#}\circ T_n^{\Delta^n}=T^X_n\circ\sigma_{\#},
\]
en particular, como $\sigma_{\#}(\Id_{\Delta})=\sigma\circ\Id=\sigma$, tenemos que
\begin{equation}\label{eq:def_de_T}
  T^X_n(\sigma)=T^X_n(\sigma_{\#}(\Id_{\Delta^n}))=
  (\sigma\times\Id_I)_{\#} \big( T^{\Delta^n}_n(\Id_{\Delta^n})\big).
\end{equation}
Por lo tanto definir $T^X_n$ se reduce a definir
$T_n^{\Delta^n}(\Id_{\Delta^n})\in S_{n+1}(\Delta^n\times I)$.

Para esto hay que hacer lo mismo que en (\ref{eq:homotopia_complejos_lambda}):
escribimos $\delta^i:\Delta^n\ra\Delta^n\times I$ como la inclusi\'on
$\delta^i(x)=(x,i)$ y queremos una familia
$\fT^{\Delta}:\{T_m^{\Delta^n}:S_m(\Delta^n)\ra S_{m+1}(\Delta^n\times I)\}$ de
morfismo de $R$-m\'odulos que sea una homotop\'ia $\delta_{\#}^0\simeq\delta_{\#}^1$,
en particular:
\begin{equation}
  \label{eq:homotopia_delta}
  \delta_{\#}^1(\Id_{\Delta^n})-\delta_{\#}^0(\Id_{\Delta^n})=
  (\partial'_{n+1}\circ T_n^{\Delta^n})(\Id_{\Delta^n})+
  (T_{n-1}^{\Delta^n}\circ\partial_n)(\Id_{\Delta^n})
\end{equation}
donde $m=n$ (ya que solamente necesitamos este caso). Primero evaluamos el segundo sumando.

Observa que los morfismos cara $F_n^i:\Delta^{n-1}\ra\Delta^n$ satisfacen el diagrama
conmutativo
\[
  \begin{tikzcd}[column sep=large]
    S_{n-1}(\Delta^{n-1}) \arrow[r,"T^{\Delta^{n-1}}_{n-1}"] \arrow[d,"f_{\#}"'] &
    S_{n}(\Delta^{n-1}\times I) \arrow[d,"(F_n^i\times\Id_I)_{\#}"]\\
    S_{n-1}(\Delta^n) \arrow[r,"T^{\Delta^n}_{n-1}"] & S_{n}(\Delta^n\times I)
  \end{tikzcd}
\]
Este diagrama nos permite calcular:
\begin{align*}
  (T_{n-1}^{\Delta^n}\circ\partial_n)(\Id_{\Delta^n}) &=
  T_{n-1}^{\Delta^n}\paren{\sum_{i=0}^n(-1)^i (\Id_{\Delta^n}\circ F_n^i)} =
  T_{n-1}^{\Delta^n}\paren{\sum_{i=0}^n(-1)^i F_n^i} =
  \sum_{i=0}^n(-1)^i(T_{n-1}^{\Delta^n}\circ F_n^i) \\ & =
  \sum_{i=0}^n(-1)^i (F_n^i\times\Id)_{\#}\big( T_{n-1}^{\Delta^{n-1}} (\Id_{\Delta^{n-1}})\big)  
\end{align*}
Sustituimos en (\ref{eq:homotopia_delta}) para obtener:
\begin{equation}
  \label{eq:homotopia_delta_2}
  \delta_{\#}^1(\Id_{\Delta^n})-\delta_{\#}^0(\Id_{\Delta^n})=
  \partial'_{n+1}\big(T_n^{\Delta^n}(\Id_{\Delta^n}) \big)+
  \sum_{i=0}^n(-1)^i (F_n^i\times\Id)_{\#}\big( T_{n-1}^{\Delta^{n-1}} (\Id_{\Delta^{n-1}})\big).
\end{equation}
Observa que el lado derecho depende de solamente $T_n^{\Delta^n}(\Id_{\Delta^n})$.
La idea es definirlo para forzar la igualdad anterior. Para esto, necesitamos una
construcci\'on nueva: agregar un v\'ertice a un simplejo singular.

\begin{defin}
  Sea $\sigma:\Delta^n\ra\RR^N$ un $n$-simplejo singular con $\sigma=\gen{v_0,\ldots,v_n}$.
  Para $v\in\RR^N$ definimos:
  \[
    v\cdot\sigma:=\gen{v,v_0,\ldots,v_n} \quad\text{donde}\quad
    (v\cdot\sigma)(e_i)=
    \begin{cases}
      v &\text{si}\;\; i=0 \\
      v_{i-1} &\text{si}\;\; i=1,\ldots,n
    \end{cases}
  \]
  En general si $\tau=\sum_{\sigma}r_{\sigma}\sigma$ es una $n$-cadena afin, entonces
  definimos
  \[
    v\cdot\tau:=\sum_{\sigma}r_{\sigma}(v\cdot\sigma).
  \]  
\end{defin}

\begin{ejemplo}
\begin{figure}[h!]%%%%%%%%%%%%%%%%%%%%%%%%%%%%%%%%%%%%%%%%%%%%%%%%%%%%%%%%%%%%%% FIGURE
  \centering
%  \includegraphics[scale=0.11]{agregar_vertice}
\end{figure}%%%%%%%%%%%%%%%%%%%%%%%%%%%%%%%%%%%%%%%%%%%%%%%%%%%%%%%%%%%%%%%%%%%%%
\end{ejemplo}

Esta construcci\'on cumple algunas propiedades importantes

\begin{lema}\label{lema:uno_invariancia}
  Las caras de $v\cdot\sigma$ son
  \[
    (v\cdot\sigma)^{(i)}=
    \begin{cases}
      \sigma &\text{si}\;\; i=0 \\
      v\cdot\sigma^{(i)} &\text{si}\;\; i=1,\ldots,n
    \end{cases}
  \]
\end{lema}
\begin{proof}
  Para $i=0$, tenemos:
  \[
    (v\cdot\sigma)^{(0)}(e_j)=(v\cdot\sigma)(F^0_{n+1}(e_j))=(v\cdot\sigma)(e_{j+1})=
    v_j=\sigma(e_j),
  \]
  por lo tanto $(v\cdot\sigma)^{(0)}=\sigma$.

  Para $i>0$ tenemos
  \begin{align*}
    (v\cdot\sigma)^{(i)}(e_j)&=(v\cdot\sigma)(F^i_{n+1}(e_j))=
    \left. \begin{cases}
      (v\cdot\sigma)(e_j) &\text{si}\;\; j<i\\
      (v\cdot\sigma)(e_{j+1}) &\text{si}\;\; j\geq i
    \end{cases}\right\}=
  \left. \begin{cases}
      v &\text{si}\;\; j=0 \\
      v_{j-1} &\text{si}\;\; 0<j<i\\
      v_j &\text{si}\;\; j\geq i
    \end{cases}\right\}\\ & =
  (v\cdot\sigma^{(i)})(e_j),
  \end{align*}
por lo tanto $(v\cdot\sigma)^{(i)}=v\cdot\sigma^{(i)}$
\end{proof}

\begin{lema}\label{lema:dos_invariancia}
  Sea $\tau=\sum_{\sigma}r_{\sigma}\sigma$ una $n$-cadena afin, entonces:
  \[
    \partial_{n+1}(v\cdot\tau)=\tau-v\cdot\partial_n(\tau),
  \]
  en particular, si $\tau$ es un ciclo entonces $\partial_{n+1}(v\cdot\tau)=\tau$.
\end{lema}
\begin{proof}
  Simplemente hay que calcular:
  \[
    \partial_{n+1}(v\cdot\tau)=\partial_{n+1}\paren{\sum_{\sigma}r_{\sigma}(v\cdot\sigma)}=
    \sum_{\sigma}r_{\sigma}\partial_{n+1}(v\cdot\tau)=
    \sum_{\sigma}r_{\sigma}\paren{\sum_{i=0}^{n+1}(-1)^i(v\cdot\sigma)^{(i)}}.
  \]
  Por el lema \ref{lema:uno_invariancia}, la suma adentro del par\'entesis es
  \[
    \sum_{i=0}^{n+1}(-1)^i(v\cdot\sigma)^{(i)}=
    \sigma-\sum_{i=0}^{n}(-1)^iv\cdot\sigma^{(i)}=
    \sigma-v\cdot\paren{\sum_{i=0}^n(-1)^i\sigma^{(i)}}=\sigma-v\cdot\partial_n(\sigma).
  \]
  Si sustituimos en la igualdad anterior, obtenemos:
  \begin{align*}
    \partial_{n+1}(v\cdot\tau) & =\sum_{\sigma}r_{sigma}(\sigma-v\cdot\partial_n(\sigma))=
    \sum_{\sigma}r_{\sigma}\sigma-\sum_{\sigma}r_{\sigma}v\cdot\partial_n(\sigma)=
    \tau-v\cdot\paren{\sum r_{\sigma}\partial_n(\sigma)}\\ & =\tau-v\cdot\partial_n(\tau).
  \end{align*}  
\end{proof}

Con esto definimos bien la familia
\[
  \sP_0:=T_0^{\Delta^0}(Id_{\Delta^0}),\;\; \sP_1:=T_1^{\Delta^1}(Id_{\Delta^1}),\; \ldots\;,
  \sP_n:=T_n^{\Delta^n}(Id_{\Delta^n}),\; \ldots
\]
de simplejos para poder definir $T^X_n$ como en (\ref{eq:def_de_T}).

\begin{defin}
  Sea $\fb_n:=(\fb(\Delta^n),\tfrac{1}{2})\in \Delta^n\times I$, donde $\fb(\Delta^n)$
  es el baricentro de $\Delta^n$. Entonces definimos inductivamente:
  \begin{equation*}
    \sP_0:=\fb_0\cdot\paren{\delta^1_{\#}(\Id_{\Delta^0})-\delta^0_{\#}(\Id_{\Delta^0})}
  \end{equation*}
  \begin{equation*}
    \sP_n:=\fb_n\cdot\underset{\fz_n}{\underbrace{\paren{\delta^1_{\#}(\Id_{\Delta^n})-
          \delta^0_{\#}(\Id_{\Delta^n})-
    \sum_{i=0}^n(-1)^i(F^i_n\times\Id_I)_{\#}(\sP_{n-1})}}}.
  \end{equation*}
\end{defin}

Ilustro los primeros simplejos:
%\begin{enumerate}
%\item[($n=0$)] Primero definimos $\sP_0:=T_0^{\Delta^0}(\Id_{\Delta^0})$. Como $\partial_0=0$,
%  la f\'ormula (\ref{eq:homotopia_delta}) se reduce a:
%  \[
%    \delta_{\#}^1(\Id_{\Delta^0})-\delta_{\#}^0(\Id_{\Delta^0})=\partial'_0(\sP_0)
%  \]
%  Entonces tenemos
\begin{figure}[h!]%%%%%%%%%%%%%%%%%%%%%%%%%%%%%%%%%%%%%%%%%%%%%%%%%%%%%%%%%%%%%% FIGURE
  \centering
%  \includegraphics[scale=0.11]{p0}
\end{figure}%%%%%%%%%%%%%%%%%%%%%%%%%%%%%%%%%%%%%%%%%%%%%%%%%%%%%%%%%%%%%%%%%%%%%

%\item[($n=1$)] Para este caso tambi\'en definimos $\sP_1:=T_1^{\Delta^1}(\Id_{\Delta^1})$,
%  la f\'ormula (\ref{eq:homotopia_delta_2}) queda:
%  \begin{align*}
%    \delta_{\#}^1(\Id_{\Delta^1})-\delta_{\#}^0(\Id_{\Delta^1})&=
%    \partial'_{2}(\sP_1)+(F_1^0\times\Id_I)_{\#}(\sP_0)-(F_1^1\times\Id_I)_{\#}(\sP_0) \\
%    \therefore \partial'_{2}(\sP_1) &= \delta_{\#}^1(\Id_{\Delta^1})-\delta_{\#}^0(\Id_{\Delta^1})+
%    (F_1^1\times\Id_I)_{\#}(\sP_0)-(F_1^0\times\Id_I)_{\#}(\sP_0)
%  \end{align*}
%  El lado derecho es la frontera de
\begin{figure}[h!]%%%%%%%%%%%%%%%%%%%%%%%%%%%%%%%%%%%%%%%%%%%%%%%%%%%%%%%%%%%%%% FIGURE
  \centering
%  \includegraphics[scale=0.11]{p1}
\end{figure}%%%%%%%%%%%%%%%%%%%%%%%%%%%%%%%%%%%%%%%%%%%%%%%%%%%%%%%%%%%%%%%%%%%%%
%
\begin{figure}[h!]%%%%%%%%%%%%%%%%%%%%%%%%%%%%%%%%%%%%%%%%%%%%%%%%%%%%%%%%%%%%%% FIGURE
  \centering
%  \includegraphics[scale=0.11]{p2}
\end{figure}%%%%%%%%%%%%%%%%%%%%%%%%%%%%%%%%%%%%%%%%%%%%%%%%%%%%%%%%%%%%%%%%%%%%%
% \end{enumerate}

Observa que $\sP_n$ se obtiene de agregarle un v\'ertice $\fb_n$, que est\'a adentro
de $\fz_n$. Intuitivamente esto produce un $(n+1)$-simplejo cuya frontera es $\fz_n$
porque todas las aristas nuevas que se producen pasan por $\fz_n$ de tal manera que
las orientaciones se cancelan. Esto es una propiedad importante que cumple la
familia $\{\sP_0,\sP_1,\ldots\}$.

\begin{lema}\label{lema:calculo_frontera_sP}
  $\partial_1(\sP_0)=\delta^1_{\#}(\Id_{\Delta^0})-\delta^0_{\#}(\Id_{\Delta^0})$ y
  en general $\partial_{n+1}(\sP_n)=\fz_n$ para $n>0$.
\end{lema}
\begin{proof}
  La prueba va a ser por inducci\'on y voy a abreviar $\delta^i=\delta^i_{\#}(\Id_{\Delta^n})$.
  Para el caso $n=0$, usamos el lema \ref{lema:dos_invariancia} para calcular:
  \[
    \partial_1(\sP_0)=\partial_n\big(\fb_0\cdot(\delta^1-\delta^0)\big)=
    \delta^1-\delta^0-\cancelto{0}{\fb_0\partial_0(\delta^1-\delta^0)}=\delta^1-\delta^0.
  \]

  En general, si $\fz_n$ es un ciclo para toda $n$, entonces el lema \ref{lema:dos_invariancia}
  nos dice que:
  \[
    \partial_{n+1}(\sP_n)=\partial_{n+1}(\fb_n\cdot\fz_n)=
    \fz_n-\cancelto{0}{\fb_0\partial_n(\fz_n)}=\fz_n.
  \]
  Por lo tanto el lema se reduce a probar que $\fz_n$ es un ciclo, ie. $\partial_n(\fz_n)=0$.

  Ahora supongamos que el lema se cumple para $n-1$, es decir $\partial_n(\sP_{n-1})=\fz_{n-1}$.
  Entonces $\partial_{n-1}(\fz_{n-1})=\partial_{n-1}(\partial_n(\sP_{n-1}))=0$. Por recursi\'on
  podemos concluir que $\fz_m$ es un ciclo para toda $m<n$. Con esto podemos calcular:
  \begin{align*}
    \partial_n(\fz_n)& =\partial_n\paren{\delta^1_{\#}(\Id_{\Delta^n})-\delta^0_{\#}(\Id_{\Delta^n})-
    \sum_{i=0}^n(-1)^i(F^i_n\times\Id_I)_{\#}(\sP_{n-1})} \\ & =
    \partial_n(\delta^1-\delta^0)-\sum_{i=0}^n(-1)^i(F^i_n\times\Id_I)_{\#}(\partial_n(\sP_{n-1}))
  \end{align*}
  donde la suma del lado derecho vale:
  \begin{align*}
  \sum_{i=0}^n(-1)^i(F^i_n\times\Id_I)_{\#}(\fz_{n-1}) & =
  \sum_{i=0}^n(-1)^i(F^i_n\times\Id_I)_{\#}\paren{%
    \delta^1-\delta^0-\sum_{j=0}^{n-1}(-1)^j(F^j_{n-1}\times\Id_I)_{\#}(\sP_{n-2})} \\ & =
  \left[ \sum_{i=0}^n(-1)^i(F^i_n\times\Id_I)_{\#}(\delta^1-\delta^0) \right]\\ & +
  \left[\sum_{i=0}^n\sum_{j=0}^{n-1}(-1)^{i+j}\big((F^i_n\circ F_{n-1}^j)\times\Id_I\big)_{\#}%
    (\partial_n(\sP_{n-1}))\right]
  \end{align*}
  Como $(F^i_n\times\Id_I)_{\#}\circ\delta^{*}=\delta^{*}\circ f_n^i$, entonces el primer
  sumando es simplemente $\partial_n(\delta^1-\delta^0)$. El segundo sumando se anula
  por el ejercicio \ref{ej:61} y por la prueba del ejercicio \ref{ej:62}. Sustituimos
  estos resultados en \ref{eq:fzn_es_ciclo} para obtener:
  \[
    \partial_n(\fz_n)=\partial_n(\delta^1-\delta^0)-\partial_n(\delta^1-\delta^0)+0=0
  \]
  y as\'i $\fz_n$ es un ciclo.
\end{proof}

Ya estamos en posici\'on para probar el teorema de la invariancia de la homologia singular;
simplemente definimos $T_n^{\Delta^n}(\Id_{\Delta^n}):=\sP_n$ como en (\ref{eq:def_de_T}).

\begin{proof}(Teorema \ref{thm:invariancia_homologia_dos})
  Recuerda que nada m\'as debemos probar que $\la^0_{\#}\simeq\la^1_{\#}$ como morfismos
  de complejos de cadenas. La homotop\'ia est\'a dada por la familia
  \[
    T_n^X:S_n(X)\lra S_{n+1}(X) \quad\text{definida por}\quad
    \sigma  \mapsto  (\sigma\times\Id_I)_{\#}(\sP_n),
  \]
  de morfismos de $R$-m\'odulos. Calculamos:
  \begin{align}\label{eq:lambdas_homotopicas}
    (\partial'_{n+1}T^X_n+T^X_{n-1}\partial_n)(\sigma)&=
    \partial'_{n+1}\big( (\sigma\times\Id_I)_{\#}(\sP_n) \big)+
    T^X_n\paren{\sum_{i=0}^n(-1)^i\sigma^{(i)}}\nonumber\\ & =
    (\sigma\times\Id_I)_{\#}\big( \partial'_{n+1}(\sP_n) \big)+
    \sum_{i=0}^n(-1)^iT^X_{n-1}(\sigma^{(i)})
  \end{align}
  Como $\partial_{n+1}'(\sP_n)=\fz_n$ por el lema \ref{lema:calculo_frontera_sP} y
  como
  \[
    T^X_{n-1}(\sigma^{(i)})=(\sigma^{(i)}\times\Id_I)_{\#}(\sP_{n-1})=
    \big((\sigma\circ F^i_n)\times\Id_I)_{\#}(\sP_{n-1})=
    \big( (\sigma\times\Id_I)\circ(F^i_n\times\Id_I) \big)(\sP_{n-1}),
  \]
  entonces (\ref{eq:lambdas_homotopicas}) se reduce a:
  \begin{align*}
    (\partial'_{n+1}T^X_n+T^X_{n-1}\partial_n)(\sigma)&=
    (\sigma\times\Id_I)_{\#}\big( \partial'_{n+1}(\sP_n) \big)+
    \sum_{i=0}^n(-1)^i\big( (\sigma\times\Id_I)\circ(F^i_n\times\Id_I) \big)(\sP_{n-1}) \\ & =
    (\sigma\times\Id_I)_{\#}\paren{\fz_n+\sum_{i=0}^n(-1)^i(F^i_n\times\Id_I)_{\#}(\sP_{n-1})} \\ & =
    (\sigma\times\Id_I)_{\#}(\delta^1-\delta^0)
  \end{align*}
  por definici\'on de $\fz_n$. Por \'ultimo 
  \[
    \big((\sigma\times\Id_I)\circ\delta^{*}\big)(v)=
    (\sigma\times\Id_I)(v,*)=(\sigma(v),*)=(\la^{*}\circ\sigma)(v)=\la^{*}_{\#}(\sigma)
  \]
  Entonces concluimos que:
  \[
    \partial'_{n+1}T^X_n+T^X_{n-1}\partial_n=\la^1_{\#}-\la^0_{\#}
  \]
  y as\'i $\la^1_{\#}\simeq\la^0_{\#}$. Por el argumento que sigue del enunciado del
  teorema \ref{thm:invariancia_homologia_dos} ya terminamos la prueba. 
\end{proof} 

\import{\directory}{ejercicios/67} %%%%%%%%%%%%%%%%%%%%%%%%%%%%%%%%%%%%%%%%%%%%%%%%%%% EJERCICIO 67

\end{document}
