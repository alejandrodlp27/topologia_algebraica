\begin{ejercicio}\label{ej:2}
  Las variedades suaves $\RR_{\iota}:=(\RR,\iota)$ y $\RR_{\Theta}:=(\RR,\Theta)$ son difeomorfos.
\end{ejercicio}

%%% RESPUESTA
\begin{proof}%  
  Defino $f:\RR\ra\RR$ y $g:\RR\ra\RR$ como $f(x)=x^{1/3}$ y $g(x)=x^3$. Observa que $(\RR,g)$ es la
  \'unica carta de $\Theta$, entonces denotar\'e por $g_{\Theta}$ como la carta y $g$ como la funci\'on
  $g:\RR\ra\RR$ apesar de que sean la misma funci\'on. Claramente $f\circ g = \Id_{\RR}=g\circ f$ y
  el diagrama
  \[
    \begin{tikzcd}
      \RR_{\iota} \arrow[rr,"f"] \arrow[d,"\Id"] &&
      \RR_{\Theta} \arrow[rr,"g"] \arrow[d,"g_{\Theta}"] &&
      \RR_{\iota} \arrow[d,"\Id"] \arrow[rr,"f"]&&
      \RR_{\Theta} \arrow[d,"g_{\Theta}"]\\
      \RR \arrow[rr,"g_{\Theta}\circ f\circ \Id^{-1}"'] &&
      \RR \arrow[rr,"\Id\circ g \circ g_{\Theta}^{-1}"'] &&
      \RR \arrow[rr,"g_{\Theta}\circ f\circ \Id^{-1}"'] &&
      \RR
    \end{tikzcd}
  \]
  es conmutativo. Los dos cambios de coordenadas son claramente suaves porque
  \[
    g\circ f \circ \Id^{-1}=\Id=\Id \circ g\circ g_{\Theta}^{-1}
  \]
  Por lo tanto $f$ y $g$ son funciones suaves tales que $f\circ g =\Id = g\circ f$ entonces
  $R_{\iota}$ y $\RR_{\Theta}$ son difeomorfos.  
\end{proof}%

