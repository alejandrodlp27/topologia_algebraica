\begin{ejercicio}\label{ej:28}
  Sea $X^n=X\times\cdots\times X$ y $G=S_n$ el grupo sim\'etrico (de permutaciones de un conjunto con
  $n$ elementos). Entonces $X^n\times S_n \ra X^n$ definido por
  \[
    \big( (x_1,\ldots,x_n),\sigma \big)\mapsto(x_{\sigma(1)},\ldots,x_{\sigma(n)})
  \]
  es una acci\'on derecha de $S_n$ sobre $X^n$.
\end{ejercicio}
%%% RESPUESTA
\begin{proof}%  
  Primero observa que el neutro $1\in S_n$ es la identidad, entonces
  \[
    (x_1,\ldots,x_n)1=(x_{1(1)},\ldots,x_{1(n)})=(x_1,\ldots,x_n).
  \]
  Ahora sea $\sigma,\tau\in S_n$. Entonces
  \begin{eqnarray*}
    (x_{1},\ldots,x_{n})(\sigma\tau) &=&
    (x_{(\sigma\tau)(1)},\ldots,x_{(\sigma\tau)(n)})=
    (x_{\sigma(\tau(1))},\ldots,x_{\sigma(\tau(n)})=
    (x_{\tau(1)},\ldots,x_{\tau(n)})\sigma \\ &=&
    ((x_{1},\ldots,x_{n})\sigma)\tau.
  \end{eqnarray*}
  Esta prueba muestra que esta acci\'on no es izquierda porque tendr\'ia
  $(\sigma\tau)x=\tau(\sigma(x))$, es decir que la segunda propiedad de acci\'on izquierda no se
  cumple porque se invierte el orden de las acciones.

  Nada m\'as falta probar que $X^n\times S_n\ra X^n$ es continua. Sea
  $U_1\times\cdots U_n\subseteq X^n$ un abierto arbitrario, es decir $U_i\subseteq X$ es abierto
  para toda $i=1,\ldots,n$. Entonces $((x_1,\ldots,x_n),\sigma)$ est\'a en la preimagen de la
  acci\'on si $x_{\sigma(i)}\in U_i$ para toda $i$, pero esto es equivalente a que
  \[
    x_i\in U_{\sigma^{-1}(i)} \qquad \forall i=1,\ldots,n.
  \]
  La equivalencia se da porque si $x_i\in U_i$ para toda $i$, entonces $x_{\tau(i)}\in U_{\tau(i)}$
  para toda permutaci\'on $\tau$; en particular $\tau=\sigma^{-1}$.
  
  Por lo tanto la preimagen de $U$ en $X\times S_n$ es:
  \[
    \bigcup_{\sigma\in S_n}
    \big((U_{\sigma^{-1}(1)}\times \cdots\times U_{\sigma^{-1}(n)})\times\{\sigma\}\big)
  \]
  que es abierto ya que $\{\sigma\}\subset S_n$ es abierto porque $S_n$ tiene la topolog\'ia
  discreta.
\end{proof}%

