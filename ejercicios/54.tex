\begin{ejercicio}\label{ej:54}
  La funci\'on inducida $\abs{f}$ est\'a bien definida, es decir para toda $\sigma\in\abs{K}$ se tiene que
  $\abs{f}(\sigma)\in\abs{L}$.
\end{ejercicio}
%%% RESPUESTA
\begin{proof}%  
	Sea $\sigma\in\abs{K}$, en particular
	\[
		\sum_{v\in V_K}\sigma(v)=1 \quad\text{y}\quad \sigma^{-1}(0,1]=\text{Sop}(\sigma)=\{v_0,\ldots,v_n\}\in K
	\]
	Como $V_K=\cup_{u\in V_L}f^{-1}[u]$ entonces:
	\begin{equation}\label{eq:primera_propiedad}
		\sum_{u\in V_L}\abs{f}(\sigma)(u)=\sum_{\tau\in L} \sum_{v\in f^{-1}[u]}\sigma(v)=\sum_{v\in V_K}\sigma(v)=1
	\end{equation}
	porque $\sigma\in\abs{K}$.

	Ahora s\'olo falta verificar que Sop$(\abs{f}(\sigma))\in L$. Como $f$ es un mapeo simplicial entonces:
	\[
		\text{Sop}(\sigma)=\{v_0,\ldots,v_n\}\in K \quad\then\quad \{f(v_0),\ldots,f(v_n)\}\in L.
	\]
	Si pruebo que Sop$(\abs{f}(\sigma))=\{f(v_0),\ldots,f(v_n)\}$ entonces por lo anterior y la f\'ormula
	(\ref{eq:primera_propiedad}) tendr\'e que $\abs{f}(\sigma)\in\abs{L}$ para toda $\sigma\in\abs{K}$.

	Pruebo la igualdad que me falta:

	$\subseteq)$ Sea $u\in\text{Sop}(\abs{f}(\sigma))\subseteq V_L$, entonces:
	\[
		0<\abs{f}(\sigma)(u)=\sum_{v\in f^{-1}[u]}\sigma(v) \quad\then\quad \exists
		v'\in f^{-1}[u] \;\;\text{tal que}\;\; \sigma(v')>0
	\]
	porque todos los sumandos son no negativos. Esto implica que $v'=v_i\in\text{Sop}(\sigma)=\{v_0,\ldots,v_n\}$
	para alguna $i$. Por lo tanto $f(v')=f(v_i)=u\in\{f(v_0),\ldots,f(v_n)\}$.

	$\supseteq)$ Considera $u\in\{f(v_0),\ldots,f(v_n)\}$. Entonces
	\[
		\abs{f}(\sigma)(u)=\abs{f}(\sigma)(f(v_i))=\sum_{v\in f^{-1}[f(v_i)]}\sigma(v)=\sigma(v_i)+\sum \sigma(v)>0
	\]
	porque $v_i\in f^{-1}[f(v_i)]$ y adem\'as $v_i\in\text{Sop}(\sigma)$ lo cual implica que $\sigma(v_i)>0$.
	Por lo tanto
	\[
		\abs{f}(\sigma)(u)>0 \quad\then\quad u\in\text{Sop}(\abs{f}(\sigma)). 
	\]
	
	Con esto conluyo que Sop$(\abs{f}(\sigma))=\{f(v_0),\ldots,f(v_n)\}$ y acabo.
\end{proof}%

