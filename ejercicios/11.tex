%%% PREGUNTA

\begin{ejercicio}\label{ej:11}
Sean $(X,x_0)$ y $(Y,y_0)$ espacios basados, entonces la funci\'on inducida por $\Phi$
del ejercicio \ref{ej:ocho}:
\[
\begin{tikzcd}
	\Big[(\Ss X,\star),(Y,y_0)\Big] \arrow[r,leftrightarrow] &
	\Big[(X,x_0),(\Omega(Y,y_0),e)\Big]
\end{tikzcd}
\]
es una biyecci\'on.
\end{ejercicio}

%%% RESPUESTA
\begin{proof}% 

Usar\'e la misma notaci\'on que el ejercicio \ref{ej:ocho} y todos los espacios son basados.

Sean $f,f'\in\text{Map}_*(\Ss X,Y)$ y $g,g'\in\text{Map}_*(X,\Omega Y)$. Si pruebo que
\[
	f\simeq f' \quad\then\quad \Phi(f)\simeq\Phi(f') \quad , \quad 
	g\simeq g' \quad\then\quad \Psi(g)\simeq\Psi(g')
\]
entonces $\Phi$ y $\Psi$ se factorizan naturalmente a trav\'es de los espacios cocientes,
es decir que existen \'unicos:
\begin{equation}\label{cd:pasar_a_cocientes}
\begin{tikzcd}
	\text{Map}_*(\Ss X, Y) \arrow[r,"\Phi"] \arrow[d,twoheadrightarrow,"\pi_1"'] & \text{Map}_*(X,\Omega Y) & &
	\text{Map}_*(X, \Omega Y) \arrow[r,"\Psi"] \arrow[d,twoheadrightarrow,"\pi_2"'] & \text{Map}_*(\Ss X,Y) \\
	\big[\Ss X, Y\big] \arrow[ur,dashed,"\bar{\Phi}"'] & & &
	\big[ X,\Omega Y \big] \arrow[ur,dashed,"\bar{\Psi}"'] &
\end{tikzcd}
\end{equation}
que hacen conmutar los diagramas, es decir que $\Phi=\bar{\Phi}\circ\pi_1$ y $\Psi=\bar{\Psi}\circ\pi_2$.
He denotado por $\pi_i$ la funci\'on cociente. Observa que por definici\'on $\bar{\Phi}([f])=\Phi(f)$.

Supongo que $f\simeq f'$ mediante la homotop\'ia $F:\Ss X\times I \ra Y$. Para un par\'ametro fijo $t\in I$
tengo que $F_t\in\text{Map}_*(\Ss X,Y)$ entonces $\Phi(F_t)\in\text{Map}_*(X,\Omega Y)$ tiene sentido y
es una funci\'on continua. Con esto defino $F':X\times I \ra \Omega Y$ con:
\[
	F'(x,t):=\Phi(F_t)(x)
\]
que es continua porque cada $\Phi(F_t)$ lo es. Observa que $F'(x,0)=\Phi(F_0)(x)=\Phi(f)(x)$ y
$F'(x,1)=\Phi(F_1)(x)=\Phi(f')(x)$ entonces $F'$ es una homotop\'ia entre $\Phi(f)$ y $\Phi(f')$, es decir
$\Phi(f)\simeq\Phi(f')$.

Por otro lado supongo que $g\simeq_G g'$ con $G:X\times I\ra \Omega Y$ una homotop\'ia. Observa que para
todo par\'ametro $G_t\in\text{Map}_*(X,\Omega Y)$ y as\'i $\Psi(G_t)\in\text{Map}_*(\Ss X,Y)$ y es continua.
Similarmente defino:
\[
	G'([x,s],t):=\Psi(G_t)[x,s]
\]
donde $[x,s]\in\Ss X$. Observa que como cada $\Psi(G_t)$ es continua entonces $G'$ tambi\'en lo es. Adem\'as
$G'([x,s],0)=\Psi(G_0)[x,s]=\Psi(g)[x,s]$ y $G'([x,s],1)=\Psi(G_1)[x,s]=\Psi(g')[x,s]$. Por lo tanto
$\Psi(g)\simeq_{G'}\Psi(g')$.

Como $\Id=\Psi\circ\Phi$, entonces:
\[
	\Phi(f)\simeq\Phi(f') \quad\then\quad \Psi(\Phi(f))=f\simeq f' =\Psi(\Phi(f'))
\]
y as\'i:
\[
	f\simeq f' \quad\iff\quad \Phi(f)\simeq\Phi(f')
\]
o equivalentement:
\[
	[f]=[f'] \quad\iff\quad \big[\Phi(f)\big]=\big[\Phi(f')\big].
\]

Si uso la notaci\'on de los diagramas conmutativos en (\ref{cd:pasar_a_cocientes}), la equivalencia
anterior quiere decir que la funci\'on
\[
	(\pi_2\circ\bar{\Phi}):[\Ss X,Y] \lra [X,\Omega Y]
%	\quad\text{y}\quad (\pi_1\circ\bar{\Psi}):[X,\Omega Y] \lra [\Ss X,Y].
\]
es inyectiva porque
\[
	(\pi_2\circ\bar{\Phi})\big( [f] \big)=
	\pi_2 \Big( \bar{\Phi} \big ( [f] \big) \Big)=
	\pi_2\big( \Phi(f) \big)=
	\big[\Phi(f)\big].
\]	
Adme\'as, si $[g]\in[X,\Omega Y]$ entonces:
\[
	(\pi_2\circ\bar{\Phi})\big( [\Psi(g)] \big)=
	\pi_2 \Big( \bar{\Phi}\big( \Psi(g) \big) \Big)=
	\pi_2 \Big( \Phi\big( \Psi(g)\big) \Big)
	\big[ \Phi(\Psi(g)) \big]=[g]
\]
y as\'i $\pi_2\circ\bar{\Phi}$ es sobreyectiva. Por lo tanto \'esta es la biyecci\'on
\[
	\big[\Ss X,Y\big] \longleftrightarrow	\big[X,\Omega Y\big].
\]
\end{proof}%

