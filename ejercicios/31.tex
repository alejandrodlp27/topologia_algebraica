\begin{ejercicio}\label{ej:31}
  $\bar{\a}_x:(G/G_x)\ra\Oo(x)$ es inyectiva. Por lo tanto es una funci\'on continua y
  biyectiva.
\end{ejercicio}
%%% RESPUESTA
\begin{proof}%  
  Como $\bar{\a}_x(gG_x)=\a_x(g)=gx$, entonces
  \[
    \bar{\a}_x(gG_x)=\bar{\a}_x(hG_x)\iff\a_x(g)=\a_x(h)\iff gx=hx \iff (gh^{-1})x=x \iff gh^{-1}\in G_x
  \]
  dice que $\a_x$ est\'a bien definida y es inyectiva. Adem\'as es sobreyectiva porque $\Oo(x)$
  se define como la imagen de $\a_x$. Por lo tanto $\a_x$ es biyectiva.

  Por \'ultimo, $\a_x=\bar{\a}_x\circ\nu$ donde $\nu:G_x\epi G/G_x$ es la identificaci\'on inducida
  por $\a_x$, entonces, como $\a_x$ es continua, $\bar{\a}_p$ es continua.
\end{proof}%

