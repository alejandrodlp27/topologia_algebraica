\begin{ejercicio}\label{ej:24}
  La relaci\'on $\sim$ definido sobre $\fB$ es de equivalencia.
\end{ejercicio}
%%% RESPUESTA
\begin{proof}%
  Primero comento que $\fB$ est\'a en biyecci\'on con las matrices invertibles: si
  $\beta=\{v_1,\ldots,v_n\}\in\fB$ con $v_j=(v_{1j},\ldots,v_{nj})\in V$, denoto por $(v_{ij})$ a
  la matriz que obtengo de tomar a $v_j$ como columnas. Por lo tanto:
  \[
    \Phi:\fB \lra \text{GL}(n,\RR)  \quad\text{con}\quad f(\beta)=(v_{ij})
  \]
  es una biyecci\'on. Por lo tanto identificar\'e $\fB$ con $G=\text{GL}(n,\RR)$ (de hecho existe
  una estructura de grupo sobre $\fB$ que hace que $\Phi$ sea un isomorfismo de grupos).
  
  Observa que $H=\text{GL}^+(n,\RR)=\{A\in G\mid \det{A}>0\}$ es un subgrupo de $G$
  entonces la funci\'on $(A,\beta)\mapsto A\beta=A(v_{ij})$ es una acci\'on de grupo.
  En efecto: $\Id\beta=\Id(v_{ij})=(v_{ij})=\beta$ y $(AB)\beta=(AB)(v_{ij})=A(B(v_{ij}))=A(B\beta)$.

  Adem\'as $\sim$ es la relaci\'on de equivalencia que define esta acci\'on: sean $\beta=(v_{ij})$ y
  $\gamma=(w_{ij})$ dos bases y $Q$ su matriz de cambio de base, ie. $Q\beta=\gamma$. Entonces:
  \[
    \beta\sim\gamma \quad\iff\quad \det Q>0 \quad\iff\quad Q\in H \quad\iff\quad \gamma\in\Oo(\beta)
  \]
  donde $\Oo(\beta)$ denota la \'orbita de $\beta$. Por lo tanto $\sim$ es una relaci\'on de
  equivalencia ya que est\'a inducida por una acci\'on de grupo.  
\end{proof}%

