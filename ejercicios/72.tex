\begin{ejercicio}\label{ej:72}
  sd${}^{\_}_n$ es una transformaci\'on natural del funtor $S_n(\_):\cat{Top}\ra{}_R\cat{Mod}$
  en si mismo.
\end{ejercicio}
%%% RESPUESTA
\begin{proof}%
  Hay que probar que el diagrama
  \[
    \begin{tikzcd}
      S_n(X) \arrow[r,"\text{sd}_n^X"] \arrow[d,"f_{\#}"] & S_n(X) \arrow[d,"f_{\#}"'] \\
      S_n(Y) \arrow[r,"\text{sd}^Y_n"] & S_n(Y)
    \end{tikzcd}
  \]
  conmutata para cualesquiera dos espacios $X$ y $Y$, y cualquier funci\'on continua
  $f:X\ra Y$ entre ellos.

  Sea $\tau=\sum_{\sigma}r_{\sigma}\sigma\in S_n(X)$, entonces:
  \begin{align*}
    f_{\#}(\text{sd}^X_n(\tau))&=
    f_{\#}\paren{\text{sd}^X_n\paren{\sum r_{\sigma}\sigma}}=
    f_{\#}\paren{\sum r_{\sigma}\text{sd}^X_n(\sigma)}=
    \sum r_{\sigma}f_{\#}\big(\sigma_{\#}(\text{sd}^{\Delta^n}_n(\Id_{\Delta^n}))\big)\\ & =
    \sum r_{\sigma}(f\circ\sigma)_{\#}(\text{sd}^{\Delta^n}_n(\Id_{\Delta^n})).
  \end{align*}
  Por otro lado:
  \begin{align*}
    \text{sd}^Y_n(f_{\#}(\tau))&=
    \text{sd}^Y_n\paren{f_{\#}\paren{\sum r_{\sigma}\sigma}}=
    \text{sd}^Y_n\paren{\sum r_{\sigma}f_{\#}(\sigma)}=
    \sum r_{\sigma}\text{sd}^Y_n(f\circ\sigma)\\ & =
    \sum r_{\sigma}(f\circ\sigma)_{\#}(\text{sd}^{\Delta^n}_n(\Id_{\Delta^n})).
  \end{align*}
  y as\'i $f_{\#}(\text{sd}^X_n(\tau))=\text{sd}^Y_n(f_{\#}(\tau))$ para toda $\tau.$ Por lo
  tanto, $\text{sd}_n^{\_}:S_n(\_)\ra S_n(\_)$ es una transformaci\'on natural.
\end{proof}%

