\begin{ejercicio}\label{ej:41}
  La funci\'on $\Phi:\eta_0\ra\ZZ_2$ definido por $[M]\mapsto \# M\mod 2$ es un isomorfismo.
\end{ejercicio}
%%% RESPUESTA
\begin{proof}%  
  Primero observa que para todo elemento $[M]\in\eta_0$, $M$ es compacto, entonces $M$ es un conjunto
  finito de puntos. As\'i tiene sentido definir $\Phi$.

  Para probar que $\Phi$ est\'a bien definido, supongo que $[M]=[N]$ para variedades cerradas
  $M=\{x_1,\ldots,x_m\}$ y $N=\{y_1,\ldots,y_n\}$. Esto quiere decir que hay una 1-variedad $W$
  tal que $\partial W= M \sqcup N$. Como $W$ es una 1-variedad, entonces necesariamente es la
  uni\'on disjunta de intervalos (ie. trayectorias) y de c\'irculos. Como $\partial\Sn^1=\emptyset$,
  puedo asumir que $W$ es simplemente la uni\'on disjunta de trayectorias.

  La frontera de una trayectoria es la uni\'on disjunta de su punto inicial y su punto final
  (ya que por suposici\'on esta trayectoria no puede ser un lazo). Por lo tanto por cada pareja
  de puntos en $M\sqcup N$ hay una trayectoria (que es una componente de $W$) que las une.
  Adem\'as esta trayectoria es \'unica ya que si hay dos trayectorias que tienen el mismo punto
  inicial (o final), este punto deja de ser parte de la frontera.

  Con esto puedo concluir que las componentes de $W$ inducen un apareamiento entre los elementos
  de $M\sqcup N$ sin dejar un punto libre (ya que de lo contrario $W$ no ser\'ia de dimensi\'on 1).
  Por lo tanto $M\sqcup N$, que tiene $n+m$ elementos, tiene una cantidad par de puntos.
  Por lo tanto $n$ y $m$ necesariamente tienen la misma paridad, o en otras palabras
  $m\equiv n\mod 2$. Con esto conluyo que $\Phi[M]=\Phi[N]$ y $\Phi$ est\'a bien definido.

  Observa que
  \[
    \Phi\big( [M]+[N]  \big)=\Phi[M\sqcup N]=\#(M\sqcup N)=\# M + \# N=\Phi[M]+\Phi[N]
  \]
  entonces al reducir m\'odulo 2 obtengo que $\Phi$ es un homomorfismo de grupos.

  Claramente es sobre porque
  \begin{eqnarray*}
    \Phi[\{x_0\}]  =  \#\{x_0\}&\equiv& 1 \mod 2 \\
    \Phi[\{x_0,x_1\}] =\#\{x_0,x_1\}= 2&\equiv& 0 \mod 2.
  \end{eqnarray*}
  Por \'ultimo, si $\Phi[M]=0$ entonces $\# M$ es par, es decir $M$ es un conjunto par de puntos,
  digamos $2n$. Si $W=I\sqcup\cdots\sqcup I$ es la uni\'on disjunta de $n$ intervalos entonces
  $M$ es difeomorfo a
  \[
    \partial W = \bigsqcup_{j=1}^n\partial I=\bigsqcup_{j=1}^n\{0,1\}
  \]
  y as\'i $M$ es la frontera de una $1$-variedad. Por lo tanto $[M]=0$ y $\ker\Phi=0$.

  Con esto he probado que $\Phi$ es un isomorfismo y que $\eta_0\cong\ZZ_2$.
\end{proof}%

