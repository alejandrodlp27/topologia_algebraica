\begin{ejercicio}\label{ej:47}
  Todo complejo simplicial geom\'etrico determina un complejo simplicial abstracto.
\end{ejercicio}
%%% RESPUESTA
\begin{proof}%  
  Sea $K=\{\sigma_1,\ldots,\sigma_n\}$ un complejo simplicial geom\'etrico con
  $\sigma_i=\gen{a_0^i,\ldots,a_{n_i}^i}$; los veo encajados en alg\'un $\RR^N$, ie. $a_j^i\in\RR^N$
  para toda $i\in\{1,\ldots,n\}$ y para toda $j\in\{0,\ldots,n_i\}$. Para cada simplejo
  $\sigma_i\in K$ defino el conjunto de sus v\'ertices como $\sigma_i'=\{a_0^i,\ldots,a_{n_i}^i\}$.

  En general escribo el conjunto de v\'ertices como
  \[
    V:=\bigcup_{i=1}^n\sigma_i'=\{a_j^i\}_{i,j}\subset\RR^N
  \]
  y defino $K'=\{\sigma_i'\}_{i=1}^n$.

  Ahora pruebo que $K'$ cumple las dos propiedades de ser un complejo simplicial abstracto.
  Primero observa que para un v\'ertice arbitrario $a_j^i\in V$, el simplejo $\gen{a_j^i}$ es
  una $0$-cara del simplejo geom\'etrico $\sigma_i=\gen{a_0^i,\ldots,a_{n_i}^i}$ entonces por
  definici\'on el 0-simplejo geom\'etrico $\gen{a_j^i}$ es un elemento de $K$, ie. $\gen{a_j^i}=\sigma_l$
  para alguna $l\in\{1,\ldots,n\}$. Por lo tanto $\sigma_l'=\{a_j^i\}\in K'$.

  Para probar la segunda propiedad sea $L=\sigma_i'=\{a_0^i,\ldots,a_{n_i}^i\}\in K'$ y sea
  $L'\subseteq L$. Entonces $L'$ genera una cara del simplejo geom\'etrico $\sigma_i$. Por definici\'on,
  el simplejo generado por $L'$ es un elemento de $K$, ie. $L$ genera a alg\'un $\sigma_k\in K$.
  Por lo tanto $L'=\sigma_k'\in K'$ y termino.
\end{proof}%

