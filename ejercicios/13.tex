%%% PREGUNTA

\begin{ejercicio}\label{ej:13}
	Sea $(Y,y_0)$ un espacio basado y $\Sn^0=\{-1,1\}$, entonces la biyecci\'on can\'onica:
	\[\begin{tikzcd}
		\text{Map}_*(\Sn^0,Y)\arrow[r,"\psi"] &  Y & \text{con} & \psi(f)=f(-1) 
        \end{tikzcd}\]
        es un homeomorfismo, ie. $\text{Map}_*(\Sn^0,Y)\approx Y.$
\end{ejercicio}

%%% RESPUESTA
\begin{proof}%
  A cada $y\in Y$ le asocio la funci\'on $f_y:\Sn^0\ra Y$ con valores $f_y(1)=y_0$ y $f_y(-1)=y$.
  A la funci\'on $y\mapsto f_y$ la denoto por $\varphi$. Demotrar\'e que $\varphi$ es la inversa
  de $\psi$ y que ambas son continuas.
  
  La primera propiedad es clara porque
  \[
    \varphi(\psi(f))(-1)=\varphi(f(-1))=f_{f(-1)}(-1)=f(-1) \quad\text{y}\quad
    \psi(\varphi(y))=\psi(f_y)=f_y(-1)=y.
  \]
  Ahora observa que para $U\subseteq Y$ abierto,
  \[
    \psi^{-1}[U]:=\{f:\Sn^0\ra Y\mid f(-1)\in U\}=
    \{f:\Sn^0\ra Y\mid f[\{-1\}] \subseteq K\}=
    B(K,U)\subset\text{Map}_*(\Sn^0,Y)
  \]
  donde $K=\{-1\}$ es un subconjunto compacto de $\Sn^0$. Es decir que $\psi^{-1}[U]$ es un abierto
  sub\'asico de la topolog\'ia compacto-abierta. Por lo tanto $\psi$ es continua.

  Adem\'as, si $B(K,U)\subseteq\text{Map}_*(\Sn^0,Y)$ es un sub\'asico. Aqu\'i $K\subseteq\Sn^0$
  es compacto, en particular es uno de los cuatro conjuntos $\emptyset$,$\{1\}$,$\{-1\},\Sn^0$.
  Tambi\'en $U\subseteq Y$ es abierto. Por definici\'on:
  \[
    \varphi^{-1}[B(K,U)]=\{y\in Y\mid f_y[K]\subseteq U\}.
  \]
  Calculo este conjunto por casos:
  
  Para $y_0\in U$, tengo
  \[
    \varphi^{-1}[B(\emptyset,U)] = Y = \varphi^{-1}[B(\{1\},U)] \quad,\quad
    \varphi^{-1}[B(\{-1\},U)] = U = \varphi^{-1}[B(\Sn^0,U)],
  \]
  y para $y_0\not\in U$, tengo
  \[
    \varphi^{-1}[B(\emptyset,U)] = Y \quad,\quad
    \varphi^{-1}[B(\{-1\},U)] = U \quad,\quad
    \varphi^{-1}[B(\{1\},U)] = \emptyset = \varphi^{-1}[B(\Sn^0,U)].
  \]
  Todos los casos dan conjuntos abiertos. Por lo tanto $\varphi^{-1}[B(K,U)]$ es abierto para
  toda $K$ y $U$, y as\'i $\varphi$ es continua. Esto termina la prueba.
\end{proof}%

