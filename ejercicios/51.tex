\begin{ejercicio}\label{ej:51}
  Prueba que solamente hay dos clases de equivalencia que son $[f]$ y $[\bar{f}]$
    donde $\bar{f}=\tau\circ f$ donde $\tau\in S_{n+1}$ es la permutaci\'on que transpone el 0
    y el 1 y fija a los dem\'as.
\end{ejercicio}
%%% RESPUESTA
\begin{proof}%  
  Sea $A_{n+1}$ el grupo alternante, ie. las permutaciones pares del grupo sim\'etrico $S_{n+1}$.
  Defino la siguiente acci\'on derecha:
  \[
    \Sigma \times A_{n+1} \lra \Sigma \quad\text{con}\quad (f,\alpha)\longmapsto f\circ\alpha.
  \]
  Claramente es una acci\'on derecha porque $(f,1)\mapsto f\circ 1=f$ y
  \begin{eqnarray*}
    ((f,\alpha),\beta)&\mapsto&(f\circ\alpha,\beta)=
    (f\circ\alpha)\circ\beta=
    f\circ(\alpha\circ\beta)\\ \therefore\;\; (f\alpha)\beta&=&f(\alpha\beta).
  \end{eqnarray*}
  Ahora sea $f\in\Sigma$. Calculo su grupo de isotrop\'ia:
  \[
    (A_{n+1})_f=\{\alpha\in A_{n+1}\mid f\circ\alpha=f\}=\{1\}.
  \]
  Esto es porque si $\alpha\in A_{n+1}$ mueve un elemento entonces $f\circ\alpha$ y $f$ difieren en
  ese elemento. M\'as precisamente, si $\alpha\neq1$, existen $i\neq j\in\underline{n}$ tal que
  $\alpha(i)=j$ y por lo tanto $(f\circ\alpha)(i)=f(\alpha(i))=f(j)\neq f(i)$, ie.
  $f\circ\alpha\neq f$, porque $f$ es biyectiva. Este argumento prueba que la acci\'on de $A_{n+1}$
  sobre $\Sigma$ es libre.

  Por el ejercicio \ref{ej:31}, hay una biyecci\'on (de hecho un homeomorfismo porque $\Sigma$ y
  $A_{n+1}$ son finitos con la topolog\'ia discreta) entre
  \[
    \Oo(f)\longleftrightarrow\frac{A_{n+1}}{(A_{n+1})_f}=\frac{A_{n+1}}{1}\cong A_{n+1} 
  \]
  Esto quiere decir que todas las \'orbita de la acci\'on tienen $(n+1)!/2$ elementos%
  \footnote{Esto se sigue del teorema de Langrange y de que $(S_{n+1}:A_{n+1})=2$. La funci\'on
    $S_{n+1}\ra\ZZ_2$ que le asocia a una permutaci\'on $\alpha\in S_{n+1}$ su signo
    $\alpha\mapsto\text{sgn}(\alpha)\in\{-1,1\}$, es un epimorfismo de grupos con kernel
    $A_{n+1}$ lo cual implica que $S_{n+1}/A_{n+1}\cong\ZZ_2$. Por el teorema de Lagrange
    $\#(S_{n+1})=(S_{n+1}:A_{n+1})\#(A_{n+1})\then\tfrac{1}{2}\#(S_{n+1})=\tfrac{1}{2}(n+1)!=
    \#(A_{n+1})$.}. %
  Adem\'as observa que:
  \[
    \Oo(f)=
    \{g\in\Sigma\mid \exists\alpha\in A_{n+1}\;\text{tal que}\; g\circ\alpha=f\}=
    \{g\in\Sigma\mid \exists \alpha\in A_{n+1}\;\text{tal que}\; \alpha=g^{-1}f\}=
    [f]\in\Sigma/_{\sim}
  \]
  

  Ahora, $\Sigma$ tiene $(n+1)!$ elementos por ser el conjunto de funciones biyectivas entre dos
  conjuntos con $n+1$ elementos. Como las \'orbitas forman una partici\'on de $\Sigma$ y cada \'orbita
  tienen la mitad de los elementos de $\Sigma$, s\'olo puede haber dos \'orbitas. Sea $f\in\Sigma$
  arbitrario, pruebo que $\Sigma/_{\sim}=\{[f],[\bar{f}]\}$, es decir que $f\not\sim\bar{f}$.
  
  Observa que:
  \[
    f\circ\bar{f}^{-1}=f\circ(\tau\circ f)^{-1}=f\circ (f^{-1}\circ\tau^{-1})=\tau^{-1}=\tau.
  \]
  Como $\tau$ es una transposici\'on, es una permutaci\'on impar. Por lo tanto $f\not\sim \bar{f}$
  y acabo. 
\end{proof}%
