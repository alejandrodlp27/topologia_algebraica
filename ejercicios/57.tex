\begin{ejercicio}\label{ej:57}
  Un morfismo $\varphi:\Cc_{\bullet}\ra\Cc_{\bullet}$ induce un homomorfismo
  $H_n(\varphi):H_n(\Cc_{\bullet})\ra H_n(\Cc'_{\bullet})$.
\end{ejercicio}
%%% RESPUESTA
\begin{proof}%  
  Sea $n\in\ZZ$ y considera el morfismo de $R$-m\'odulos $\varphi_n:\Cc_n\ra\Cc'_n$ que da la definici\'on de
  $\varphi:\Cc_{\bullet}\ra\Cc'_{\bullet}$. Tambi\'en sea $x\in\ker\partial_n\subseteq C_n$. Como $\varphi$ es
  un morfismo de cadenas, entonces:
  \[
    0=\varphi_{n-1}(0)=\varphi_{n-1}(\partial_n(x))=\partial'_n(\varphi_n(x)) \quad\then\quad \varphi_n(x)\in\ker\partial'_n.
  \]
  Esto implica que $\varphi_n|_{\ker\partial_n}:\ker\partial_n\ra\ker\partial'_n$ est\'a bien definida.

  Ahora considera $[x],[x']\in H_n(\Cc_{\bullet})$ tales que $[x]=[x']$ o equivalentemente $x-x'\in\text{Im}\partial_{n+1}$.
  Entonces existe una $y\in C_{n+1}$ tal que $\partial_{n+1}(y)=x-x'$. Como $\varphi$ es un morfismo de complejos
  de cadena, tenemos que
  \[
    \varphi_n(x)-\varphi_n(x')=\varphi_n(x-x')=\varphi_n(\partial_{n+1}y)=\partial'_{n+1}(\varphi_{n+1}(y)) \quad\then\quad
    \varphi_n(x)-\varphi_n(x')\in\text{Im}\partial'_{n+1}.
  \]
  Por lo tanto $[\varphi_n(x)]=[\varphi_n(x')]$ en $H_n(\Cc'_{\bullet})$ y as\'i la funci\'on:
  \[
    H_n(\varphi):H_n(\Cc_{\bullet})\lra H_n(\Cc'_{\bullet}) \quad\text{definido por}\quad [x] \mapsto [\varphi_n(x)]
  \]
  est\'a bien definida. Como $\varphi_n$ es un morfismo de $R$-m\'odulos, entonces $H_n(\varphi)$ es un morfismo de
  $R$-m\'odulos gracias a la regla de correspondencia de $H_n(\varphi)$.
\end{proof}%

