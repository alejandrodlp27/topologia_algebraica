\begin{ejercicio}\label{ej:19}
  Los atlas $\iota$ y $\Theta$ de $\RR$ no son compatibles, es decir alg\'un cambio de coordenadas
  no es una funci\'on suave.
\end{ejercicio}

%%% RESPUESTA
\begin{proof}%  
  Denoto $f$ como la funci\'on $x\mapsto x^3$, es decir $(\RR,f)$ es la carta de $\Theta$.
  Ahora, $f^{-1}$ no es una funci\'on suave en $x=0$ porque su derivada:
  \[
    (f^{-1})'(x)=\frac{1}{3}x^{-\frac{2}{3}}
  \]
  no est\'a bien definida en $x=0$ porque se hace arbitrariamente grande alrededor del 0, es decir
  $(f^{-1})'(x)\ra\infty$ cuando $x\ra 0$. Por lo tanto el cambio de coordenadas
  $\Id_{\RR}\circ f^{-1}=f^{-1}$, que est\'a definido sobre $f^{-1}[\RR\cap\RR]=\RR$, no es suave
  sobre todo su dominio.

  Esto quiere decir que la carta $(\RR,f)$ no est\'a en el atlas maximal generado por el atlas
  can\'onico $\iota$ y as\'i las estructuras diferenciables de las variedades $(\RR,\iota)$ y
  $(\RR,\Theta)$ no son la misma.  
\end{proof}%



