\begin{ejercicio}\label{ej:46}
  La funci\'on $f:\sigma\ra\tau$ definida por $f(\sum\la_ia_i)=\sum\la_ib_i$ es un homeomorfismo.%
  \footnote{En clase probamos esto usando el hecho que los $n$-simplejos geom\'etricos son compactos
    (cf. ejercicio \ref{ej:45}), pero modifiqu\'e esta prueba para no usar la compacidad para poder
    demostrar el ejercicio \ref{ej:45} con este ejercicio y as\'i evitar un c\'irculo l\'ogico.}
\end{ejercicio}
%%% RESPUESTA
\begin{proof}%
  Primero observa que si $x=\sum\la_ia_i\in\sigma$, entonces $\la_0=1-\la_1-\cdots-\la_n$ y as\'i:
  \begin{eqnarray*}
    x&=&
    \la_0a_0+\sum_{i=1}^n\la_ia_i=
      a_0(1+\sum_{i=1}^n(-\la_i))+\sum_{i=1}^n\la_ia_i=
      a_0+\sum_{i=1}^n\la_i(a_i-a_0)\\
    \therefore x-a_0 &=& \sum_{i=1}^n\la_i(a_i-a_0)
  \end{eqnarray*}
  
  Ahora defino $T_{a_0}:\RR^{n}\ra\RR^{n}$ como la traslaci\'on $x\mapsto x-a_0$. Por lo tanto
  $T_{a_0}[\sigma]=\gen{0,a_1-a_0,\ldots,a_n-a_0}$ donde $A=\{a_1-a_0,\ldots,a_n-a_0\}$ es una base.
  Despu\'es defino $F:\RR^n\ra\RR^n$ como la transformaci\'on lineal que cambia la base $A$ en la
  base $B=\{b_1-b_0,\ldots,b_n-b_0\}$, en particular $F(a_i-a_0)=b_i-b_0$. Por \'ultimo sea
  $T_{-b_0}:\RR^n\ra\RR^n$ la traslaci\'on $x\mapsto x+b_0$. Claramente cada una de estas tres funciones
  son homeomorfismos entonces su composici\'on es un homeomorfismo de $\RR^{n}$ en $\RR^n$.

  Para $x=\sum\la_ia_i\in\sigma$ calculo:
  \begin{eqnarray*}
    (T_{-b_0}\circ F\circ T_{a_0})(x)&=&
    (T_{-b_0}\circ F)(x-a_0)=
    (T_{-b_0}\circ F)\paren{\sum_{i=1}^n\la_i(a_i-a_0)}\\ &=&
    T_{-b_0}\paren{\sum_{i=1}^n\la_i(b_i-b_0)}=
    b_0+\sum_{i=1}^n\la_i(b_i-b_0)=
    \sum_{i=0}^n\la_ib_i=f(x)\\
    \therefore (T_{-b_0}\circ F\circ T_{a_0})|_{\sigma}&=&f
  \end{eqnarray*}
  y as\'i $f$ es casi un homeomorfismo, lo \'unico que hace falta calcular es la imagen de
  $f$. Pero esto es sencillo:
  \[
    \text{Im}(f)=
    \{f(\sum\la_ia_i)\mid \sum\la_ia_i\in\sigma\}=
    \{\sum \la_ib_i\mid \la_i\geq 0,\;\; \sum\la_i=1\}=
    \tau.
  \]
  Por lo tanto:
  \[
    (T_{-b_0}\circ F\circ T_{a_0})|_{\sigma}=f:\sigma\ra\tau
  \]
  es un homeomorfismo por ser restricci\'on del homeomorfismo $T_{-b_0}\circ F\circ T_{a_0}$
  (co-restringido a su imagen).
\end{proof}%

