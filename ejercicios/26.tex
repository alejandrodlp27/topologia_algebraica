\begin{ejercicio}\label{ej:26}
  Para toda $x\in\Sn^1$, existe una vecindad abierta $U\subseteq\Sn^1$ de $x$ tal que
  $\epsilon^{-1}[U]=\bigcup U_n$ con $U_n\cap U_m=\emptyset$ donde $\epsilon_n:=\epsilon|_{U_n}:U_n\ra U$
  es un homeomorfismo.
\end{ejercicio}
%%% RESPUESTA
\begin{proof}%  
  Veo a $\Sn^1$ encajado en el plano complejo y sea $e^{\theta}\in\Sn^1$. Sin p\'erdida de generalidad
  puedo asumir que $e^{\theta}\neq -1$ porque de otra manera nada m\'as encajo $\Sn^1$ como
  $-\Sn^1\subset\CC$.

  Observa que $e^{\theta}\in\CC-\{x\in\RR \mid x<0\}$ entonces la rama principal del argumento est\'a
  bien definida y $\arg(e^{\theta})=\vartheta\in(-\pi,\pi)$ donde $\theta=\vartheta+2\pi n$ para alguna
  $n\in\ZZ$ (aqu\'i $\arg$ es una funci\'on continua).

  De esta manera para cada $z\in\Sn^1-\{-1\}$ hay un \'unico $\vartheta\in(-\pi,\pi)$ tal que
  $z=e^{\vartheta}$, ie. $\arg(z)=\vartheta$. Con esto puedo definir una funci\'on continua
  $\delta:\Sn^1-\{-1\}\ra (-\tfrac{1}{2},\tfrac{1}{2})$ como
  \[
    \delta(z)=\delta(e^{\vartheta})=\frac{\vartheta}{2\pi}.
  \]
  Claramente es continua y adem\'as es el inverso de la restricci\'on
  $\bar{\epsilon}:=\epsilon|_{(-\tfrac{1}{2},\tfrac{1}{2})}$ porque si
  $t\in(-\tfrac{1}{2},\tfrac{1}{2})$,
  o equivalentemente $2\pi t\in(-\pi,\pi)$, y si $z=e^{\vartheta}\in\Sn^1-\{-1\}$ entonces:
  \[
    \delta(\bar{\epsilon}(t))=
    \delta(e^{2\pi t})=
    \frac{2\pi t}{2\pi}=
    t \quad\text{y}\quad
    \bar{\epsilon}(\delta(e^{\vartheta}))=
    \bar{\epsilon}\paren{\tfrac{\vartheta}{2\pi}}=
    e^{2\pi\tfrac{\vartheta}{2\pi}}=
    e^{\vartheta}.
  \]
  Por lo tanto $\Sn^1-\{-1\}\approx(-\tfrac{1}{2},\tfrac{1}{2})$ mediante la restricci\'on de $\epsilon$.

  Ahora escribo $U=\Sn^1-\{-1\}$ como la vecindad abierta de alg\'un elemento $z\neq -1$ arbitrario de
  $\Sn^1$. Observa que
  \[
    \epsilon^{-1}[U]=
    \{t\in\RR \mid e^{2\pi t}\neq -1\}=
    \{t\in\RR \mid t\not\in\paren{\tfrac{1}{2}+\ZZ}\}=
    \bigcup_{n\in\ZZ}\paren{n-\tfrac{1}{2},n+\tfrac{1}{2}}
  \]
  donde $\tfrac{1}{2}+\ZZ=\{\tfrac{1}{2}+n\in\RR\mid n\in\ZZ\}$; escribo
  $U_n:=\paren{-\tfrac{1}{2}+n,\tfrac{1}{2}+n}$. Claramente las $U_n$'s son disjuntas porque cada
  una pertenece a un elemento distinto de la partici\'on
  $\{ [n-\tfrac{1}{2},n+\tfrac{1}{2})\}_{n\in\ZZ}$ de $\RR$.

  Adem\'as sea
  $\varphi_{-n}:U_n\ra(-\tfrac{1}{2},\tfrac{1}{2})$ la traslaci\'on $\varphi_{-n}(t)=t-n$. Claramente
  es un homeomorfismo porque es la restricci\'on de un homeomorfismo con el codominio igual a su imagen.
  Por lo tanto
  \[
    \epsilon_n:=\epsilon|_{U_n}=\bar{\epsilon}\circ\varphi_{-n} \quad\text{porque}\quad
    \bar{\epsilon}(\varphi_{-n}(t))=\bar{\epsilon}(t-n)=e^{2\pi(t-n)}=e^{2\pi t}=\epsilon_n(t).
  \]
  Esto quiere decir que $\epsilon_n:U_n\ra U$ es un homeomorfismo por ser la composici\'on de
  homeomorfismos.  
\end{proof}%

