\begin{ejercicio}\label{ej:53}
	Para toda $n\geq 1$, se cumple que $\partial_{n-1}\circ\partial_{n}=0$, en particular Im$(\partial_n)\subseteq\ker(\partial_{n-1})$.
\end{ejercicio}
%%% RESPUESTA
\begin{proof}%  
	Sea $[v_0,\ldots,v_n]\in S_n^{\fO}$. Entonces:
	\begin{equation}\label{eq:primera_formula}
		\partial_{n-1}(\partial_n[v_0,\ldots,v_n])=
		\partial_{n-1}\paren{\sum_{i=0}^n(-1)^i [v_0,\ldots,\what{v_i},\ldots,v_n]}=
		\sum_{i=0}^n(-1)^i \partial_{n-1}[v_0,\ldots,\what{v_i},\ldots,v_n]
	\end{equation}
	Ahora analizo el t\'ermino $\partial_{n-1}[v_0,\ldots,\what{v_i},\ldots,v_n]$. Observa que
	\begin{align*}
		\partial_{n-1}[v_0,\ldots,\what{v_i},\ldots,v_n] &=
		[\what{v_0},\ldots,\what{v_i},\ldots,v_n]+\cdots+(-1)^{i-1}[v_0,\ldots,\what{v_{i-1}},\what{v_i},\ldots,v_n] \\ &+
		(-1)^i[v_0,\ldots,\what{v_{i}},\what{v_{i+1}},\ldots,v_n]+\cdots+(-1)^{n-1}[v_0,\ldots,\what{v_i},\ldots,\what{v_n}]
	\end{align*}
	Juntando esto en notaci\'on de suma, tenemos:
	\[
		\partial_{n-1}[v_0,\ldots,\what{v_i},\ldots,v_n] =
		\sum_{j=0}^{i-1}(-1)^j[v_0,\ldots,\what{v_j},\ldots,\what{v_i},\ldots,v_n]+\sum_{k=i+1}^{n}(-1)^{k-1}[v_0,\ldots,\what{v_i},\ldots,\what{v_k},\ldots,v_n]
	\]
	donde el exponente de $(-1)$ en la segunda suma es $k-1$ en lugar de $k$ porque el $k$-\'esimo sumando de la
	segunda suma es el $(k-1)$-\'esimo sumando de todo $\partial_{n-1}[v_0,\ldots,\what{v_i},\ldots,v_n]$ (ya que $k>i$
	y se elimin\'o el t\'ermino $i$-\'esimo). Sustituyo en la f\'ormula (\ref{eq:primera_formula}):
	\begin{align*}
		\partial_{n-1}(\partial_n[v_0,\ldots,v_n]) &=
		\sum_{i=0}^n(-1)^i \left( \sum_{j=0}^{i-1}(-1)^i[v_0,\ldots,\what{v_j},\ldots,\what{v_i},\ldots,v_n] \right. \\ &
		\left. +\sum_{k=i+1}^{n}(-1)^{k-1}[v_0,\ldots,\what{v_i},\ldots,\what{v_k},\ldots,v_n]\right) \\ & =
		\sum_{i=0}^n\sum_{j=0}^{i-1}(-1)^{i+j}[v_0,\ldots,\what{v_j},\ldots,\what{v_i},\ldots,v_n] \\ & +
		\sum_{i=0}^n\sum_{j=i+1}^{n}(-1)^{i+j-1}[v_0,\ldots,\what{v_i},\ldots,\what{v_j},\ldots,v_n] 
	\end{align*}
	De esta expressi\'on se vuelve claro que un sumando arbitrario $[v_0,\ldots,\what{v_k},\ldots,\what{v_l},\ldots,v_n]$
	aparece dos veces con signo contrario: en el primer sumando cuando $k=j$ y $l=i$ con signo $(-1)^{i+j}$ y en el
	segundo sumando cuando $k=i$ y $l=j$ con signo $(-1)^{i+j-1}$.

	Por lo tanto cada sumando de la primera suma, de la ecuaci\'on anterior, se cancela con un sumando de la segunda
	suma. As\'i podemos concluir que toda la suma vale 0, ie. $\partial_{n-1}(\partial_n[v_0,\ldots,v_n])=0$ para todo
	elemento de $S_n^{\fO}$.

	Ahora $\partial_{n-1}\circ\partial$ se extiende a $R\gen{S_n^{\fO}}$ como:
	\[
		(\partial_{n-1}\circ\partial_n)\paren{\sum_{\sigma\in S_n^{\fO}} r_{\sigma}\sigma}=
		\sum r_{\sigma} (\partial_{n-1}\circ\partial_n)(\sigma)=
		\sum r_{\sigma} 0=0
	\]
	y as\'i podemos concluir que $\partial_{n-1}\circ\partial_n=0$ como morfismo de $R$-m\'odulos.
\end{proof}%

