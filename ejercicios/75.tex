\begin{ejercicio}\label{ej:75}
  Las tres definiciones de homolog\'ia reducida son equivalentes.
\end{ejercicio}
%%% RESPUESTA
\begin{proof}%
  El teorema \ref{thm:sucesion_exacta_larga} dice que existe una sucesi\'on exacta larga:
  \[
    \begin{tikzcd}
      \cdots \arrow[r] & H_n(\{x\}) \arrow[r] & H_n(X) \arrow[r] & H_n(X,\{x\})\arrow[r,"d_n"] &
      H_{n-1}(\{x\}) \arrow[r] & H_{n-1}(X) \arrow[r] & \cdots.
    \end{tikzcd}
  \]
  Pero ya hemos calculado la homolog\'ia de un punto (cf. proposici\'on \ref{prop:homologia_punto}),
  entonces la sucesi\'on exacta larga se reduce a:
  \[
    \begin{tikzcd}
      \cdots \arrow[r] & 0 \arrow[r] & H_n(X) \arrow[r] & H_n(X,\{x\})\arrow[r,"d_n"] &
      0 \arrow[r] & \cdots
    \end{tikzcd}
  \]
  y para $n=0,1$ la sucesi\'on exacta larga termina en
  \[
    \begin{tikzcd}
      \cdots \arrow[r] & 0 \arrow[r] & H_1(X) \arrow[r]&
      H_1(X,\{x\}) \arrow[r,"d_1"] & R \arrow[r] & H_0(X) \arrow[r] & H_0(X,\{x\}) \arrow[r] & 0.
    \end{tikzcd}
  \]
  Para $n>1$ podemos concluir que $H_n(X)\cong H_n(X,\{x\})$
\end{proof}%

