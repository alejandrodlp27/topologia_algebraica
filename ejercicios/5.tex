%%% PREGUNTA

\begin{ejercicio}
Sea $X$ un espacio topol\'ogico y $x_0,x_1\in X$ tales que existe una trayectoria $\sigma:I\ra X$
tal que $\sigma(0)=x_0$ y $\sigma(1)=x_1$. Entonces:
\[
	\pi_1(X,x_0) \cong \pi_1(X,x_1)
\]
mediante el ismorfismo $\hat{\sigma}\big([\alpha]\big)=[\bar{\sigma}*\alpha*\sigma]$
\end{ejercicio}

%%% RESPUESTA
\begin{proof}% 

Primero verifico que $\hat{\sigma}$ est\'a bien definida. Observemos que, como $\sigma$ es continua, 
entonces $\bar{\sigma}*\alpha*\sigma$ tambi\'en lo es. Observa que
\[
	(\bar{\sigma}*\alpha*\sigma)(0)=\bar{\sigma}(0)=x_1=\sigma(1)=(\bar{\sigma}*\alpha*\sigma)(1).
\]
Por lo tanto $\hat{\sigma}([\alpha])\in\pi_1(X,x_1)$. Adem\'as, como $\hat{\sigma}$ est\'a definida
para clases de equivalencia hay que probar que respeta homotop\'ias:

Sean $\alpha\simeq \beta$ lazos en $(X,x_0)$. Entonces $\alpha*\sigma\simeq \beta*\sigma$. Esto se
parece mucho al resultado que us\'e para probar que la operaci\'on $[\alpha][\beta]=[\alpha*\beta]$
est\'a bien definida; la diferencia es que $\sigma$ es una trayectoria en lugar de un lazo. Sin
embargo la homotop\'ia es la misma:

Si $\alpha\simeq_H\beta$ (relativo a $A=\{0,1\}$) entonces $\alpha*\sigma\simeq_F\beta*\sigma$ donde
\[
	F(s,t):=
	\begin{cases}
		H(2s,t) & \text{si}\;\; 0\leq s\leq \frac{1}{2} \\
		\sigma(2s-1) & \text{si}\;\; \frac{1}{2}\leq s\leq 1
	\end{cases}.
\]

Volvemos a aplicar este resultado a $(\alpha*\sigma)$ y concluimos que
$\bar{\sigma}*\alpha*\sigma\simeq \bar{\sigma}*\beta*\sigma$. Por lo tanto
\[
	\hat{\sigma}\big([\alpha]\big)=[\bar{\sigma}*\alpha*\sigma]=\bar{\sigma}*\beta*\sigma=
	\hat{\sigma}\big([\beta]\big)
\]
cuando $\alpha\simeq\beta$ y as\'i $\tilde{\sigma}$ est\'a bien definido.

Para probar que es un homomorfismo de grupos, hay que probar la siguiente igualdad:
\[
	\hat{\sigma}\big( [\alpha][\beta] \big)=\hat{\sigma}\big( [\alpha]\big)\hat{\sigma}\big([\beta]\big).  
\]
Aplico la asociatividad del grupo fundamental y la existencia de neutros para concluir que:
\begin{align*}%
	\hat{\sigma}\big( [\alpha][\beta] \big)=\hat{\sigma}\big( [\alpha*\beta] \big) & =%
	[\bar{\sigma}*\alpha*\beta*\sigma]=[\bar{\sigma}][\alpha][\beta][\sigma]=%
	[\bar{\sigma}][\alpha]\Big([\sigma][\sigma]^{-1} \Big)[\beta][\sigma]\\ & =%
	[\bar{\sigma}][\alpha]\Big([\sigma][\bar{\sigma}] \Big)[\beta][\sigma]=%
	\Big( [\bar{\sigma}][\alpha][\sigma] \Big) \Big( [\bar{\sigma}][\beta][\sigma] \Big) \\ & =%
	\hat{\sigma}\big( [\alpha]\big)\hat{\sigma}\big([\beta]\big).%
\end{align*}%


\end{proof}%

