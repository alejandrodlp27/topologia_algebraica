\begin{ejercicio}\label{ej:33}
  Sea $p:E\ra X$ un cubriente. Si $E$ es conectable por trayectorias entonces la acci\'on
  $E_x\times\pi(X)\ra E_x$ es transitiva.
\end{ejercicio}
%%% RESPUESTA
\begin{proof}%  
  Sean $e,e'\in E_x$. Como $E$ es conectable por trayectorias, existe un $\tau:I\ra E$ tal que
  $\tau(0)=e$ y $\tau(1)=e'$. Entonces $\sigma:=p\circ\tau$ es un lazo en $X$ porque
  \[
    \sigma(0)=p(\tau(0))=p(e)=x=p(e')=p(\tau(1))=\sigma(1).
  \]
  Por lo tanto si elijo $e\in E$, el lazo $\sigma$ se levanta a una \'unica trayectoria
  $\what{\sigma}_e$ tal que $\what{\sigma}_e(0)=e$ y $p\circ\what{\sigma}_e=\sigma$. Pero
  $\tau$ tambi\'en cumple estas propiedades ($\sigma=p\circ\tau$ por definici\'on y $\tau(0)=e$).
  Por la unicidad del levantamiento tengo que $\what{\sigma}_e=\tau$, por lo tanto:
  \[
    e[p\circ\tau]=e[\sigma]=\what{\sigma}_e(1)=\tau(1)=e'
  \]
  y la acci\'on es transitiva.   
\end{proof}%

