\begin{ejercicio}\label{ej:34}
  Sean $p:E\ra X$ un cubriente y $p_{\#}:\pi_1(E,e)\ra\pi_1(X,x)$ el homomorfismo inducido donde
  $x\in X$ es fija y $e\in E_x$. Si denoto $G=\pi_1(X,x)$, el grupo de isotrop\'ia de $e$ bajo la
  acci\'on inducida por el cubriente es:
  \[
    G_e=p_{\#}\big[\pi_1(E,e)\big].
  \]  
\end{ejercicio}
%%% RESPUESTA
\begin{proof}$\;$\\ 
\begin{enumerate}
\item[($\subseteq$)] Sea $[\sigma]\in G_e$ con $\sigma:I\ra X$ un lazo y $\what{\sigma}_e:I\ra E$
  su levantamiento, en particular $\what{\sigma}_e(0)=e$. Como $[\sigma]\in G_e$, entonces:
  \[
    e[\sigma]=\what{\sigma}_e(1)=e
  \]
  y as\'i $\what{\sigma}_e$ es un lazo en $E$, ie. $[\what{\sigma}_e]\in\pi_1(E)$. por \'ultimo,
  como $\what{\sigma}_e$ es el levantamiento de $\sigma$, entonces $p\circ\what{\sigma}_e=\sigma$
  y as\'i:
  \[
    p_{\#}[\what{\sigma}_e]=[p\circ\what{\sigma}_e]=[\sigma].
  \]
  Por lo tanto $[\sigma]\in p_{\#}[\pi_1(E)]$.
  
\item[($\supseteq$)] Sea $p_{\#}[\sigma]=[p\circ\sigma]\in p_{\#}[\pi_1(E)]$ donde
  $[\sigma]\in\pi_1(E,e)$, ie. $\sigma(0)=e=\sigma(1)$. Si escribo $\tau:=p\circ\sigma$ para
  el lazo en $X$, su levantamiento es $\what{\tau}_e$ donde $p\circ\what{\tau}_e=\tau$ y
  $\what{\tau}_e(0)=e$. Observa que $\sigma$ tambi\'en cumple estas dos propiedades entonces
  la unicidad del levantamiento dice que $\what{\tau}_e=\sigma$. Por lo tanto:
  \[
    e[p\circ\sigma]=e[\tau]=\what{\tau}_e(1)=\sigma(1)=e \quad\then\quad
    [p\circ\sigma]\in G_e.
  \]
  
\end{enumerate}
\end{proof}%

