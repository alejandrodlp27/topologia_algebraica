\begin{ejercicio}\label{ej:74}
  El morfismo $t_{\bullet}:S_{\bullet}(X)\ra S_{\bullet}^{\fU}(X)$ pasa al cociente:
  \[
    \overline{t_{\bullet}}:
    \frac{S_{\bullet}(X)}{S_{\bullet}(A)}\lra\frac{S_{\bullet}^{\fU}(X)}{S_{\bullet}(A)}
  \]
  y es una equivalencia homot\'opica de complejos de cadena, m\'as precisamente
  $\bar{\imath}\circ\overline{t_{\bullet}}\simeq\Id\simeq \overline{t_{\bullet}}\circ\bar{\imath}$.
\end{ejercicio}
%%% RESPUESTA
\begin{proof}%  
  Como $t_{\bullet}|_{S_{\bullet}^{\fU}(X)}=\Id$ y
  $S_{\bullet}(A)\subseteq S_{\bullet}(X-U)+S_{\bullet}=S_{\bullet}^{\fU}(X)$, entonces
  \[
    t_{\bullet}[S_{\bullet}(A)]=S_{\bullet}(A)
  \]
  y as\'i $t_{\bullet}$ pasa al cociente, es decir existe un \'unico $\overline{t_{\bullet}}$
  tal que:
  \[
    \begin{tikzcd}
      S_{\bullet}(X) \arrow[r,"t_{\bullet}"] \arrow[d,two heads] &
      S_{\bullet}^{\fU}(X) \arrow[r,two heads] & \tfrac{S_{\bullet}^{\fU}(X)}{S_{\bullet}(A)} \\
      \tfrac{S_{\bullet}(X)}{S_{\bullet}(A)} \arrow[urr,dashed ,"\overline{t_{\bullet}}"'] &&
    \end{tikzcd}
  \]

  Como
  \[
    t_{\bullet}|_{S_{\bullet}^{\fU}(X)}=\Id \quad\then\quad
    \imath\circ t_{\bullet}=\Id \quad\then\quad
    \bar{\imath}\circ\overline{t_{\bullet}}=\Id
  \]
  y as\'i tenemos trivialmente que $\Id\simeq\bar{\imath}\circ \overline{t_{\bullet}}$. Para
  probar el inverso, hay que dar una homotop\'ia de complejos de cadena entre $\overline{t_{\bullet}}$ y
  $\bar{\imath}$; esta homotop\'ia va a ser la inducida por $\bar{\fR}=\{\overline{R_n}\}_{n\in\ZZ}$
  donde $\fR=\{R_n\}$ es la hommotop\'ia $t_{\bullet}\simeq\Id$.

  De hecho, si probamos que las $R_n:S_n(X)\ra S_n(X)$ pasan al cociente entonces:
  \begin{align*}
    (\overline{R_{n-1}}\partial_n+\partial_{n+1}\overline{R_n})[\sigma] & =
    \overline{R_{n-1}}[\partial_n(\sigma)]+\partial_{n+1}[R_n(\sigma)]=
    [R_{n-1}(\partial_n(\sigma))]+[\partial_{n+1}R_n(\sigma)]\\ & =
    [\sigma-t_n(\sigma)]=[\sigma]-[t_n(\sigma)]=\Id(\sigma)-\overline{t_n}(\sigma) \\
    \therefore \Id \simeq \Id\circ\overline{t_{\bullet}}
  \end{align*}
  porque $t_{n}(\sigma)=\sigma+\partial_{n+1}R_n(\sigma)+R_{n-1}\partial_n(\sigma)$ por
  definici\'on.
\end{proof}%

