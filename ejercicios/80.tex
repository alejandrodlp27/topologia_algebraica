\begin{ejercicio}\label{ej:80}
  Sea $n=2m+1$ impar y define $v:\Sn^n\ra\RR^{n+1}$ como
  \[
    v(x_0,x_1,\ldots,x_{2m+1})=(-x_1,x_0,-x_3,x_2,-x_5,x_4,\ldots,-x_{2m+1},x_{2m}).
  \]
  Prueba que el campo $V$ no se anula y es tangente a la esfera.
\end{ejercicio}
%%% RESPUESTA
\begin{proof}%  
  Observa que
  \[
    \|v(x)\|^2=\sum_{i\;\text{par}}x_i^2+\sum_{i\;\text{impar}}(-x_i)^2=\sum_{i=0}^{2m+1}x_i^2=\|x\|^2.
  \]
  Por lo tanto
  \[
    v(x)=0 \quad\iff\quad 0=\|v(x)\|=\|x\| \quad\iff\quad x=0,
  \]
  pero esto no sucede para $x\in\Sn^{2m+1}$. Por lo tanto $v:\Sn^{n}\ra\RR^{n+1}$ no se
  anula.

  Por \'ultimo hay que verificar que $v(x)\in T_x\Sn^n=\{y\in\RR^{n+1}\mid\gen{x,y}=0\}$.
  Si divides los sumandos de $\gen{x,v(x)}$ en grupos de dos es f\'acil calcular la suma:
  \[
    \gen{x,v(x)}=\sum_{i=0}^{m}\big(x_{2i}(-x_{2i+1})+x_{2i+1}x_{2i}\big)=0.
  \] 
\end{proof}%

